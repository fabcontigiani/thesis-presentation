\documentclass[10pt]{beamer}
\usepackage[spanish]{babel}
\usepackage{csquotes}
\usepackage[T1]{fontenc}

\setbeameroption{hide notes} % Only slides
%\setbeameroption{show only notes} % Only notes
% \setbeameroption{show notes on second screen=bottom} % Both

\usepackage{graphicx}

\usepackage{colortbl}

\usetheme{moloch}
\molochset{block=fill}

\setbeamertemplate{page number in head/foot}[appendixframenumber]
\setbeamertemplate{section in toc}[sections numbered]

\usepackage{plex-serif}
\usepackage{plex-sans}
\usepackage{plex-mono}

\usepackage{hyperref}
\hypersetup{
    colorlinks=false,
    pdftitle={Sistema de Monitoreo y Alerta Temprana basado en IA para Áreas Protegidas}
}

% Bibliografia
\usepackage[style=ieee, backend=biber]{biblatex}
\addbibresource{src/bachelor-thesis/referencias.bib}

\title{Sistema de Monitoreo y Alerta Temprana basado en IA para Áreas Protegidas}
\subtitle{Proyecto Final Integrador - Ingeniería en Computación}
\author{Fabrizio Contigiani - Gabriel Da Silva}
\institute{Tutor: Dr. Ing. Sergio Moya} % institute already in titlegraphic
\date{\today}
\titlegraphic{\includegraphics[width=\textwidth]{img/logos.png}}

\begin{document}

\maketitle

\begin{frame}
	\frametitle{Índice}
	\tableofcontents[hideallsubsections]
\end{frame}

% ==============================================================================
% Sección 1: Contexto Regional
% ==============================================================================
\section{Contexto Regional}

\begin{frame}{El Bosque Atlántico}
	\begin{columns}
		\begin{column}{0.5\textwidth}
			\begin{itemize}
				\item Originalmente: 1.3 millones de km² \cite{ribeiro2009atlantic}
				\item Brasil (92\%), Paraguay (6\%), Argentina (2\%)
				\item Hoy: solo 12-17\% de su extensión original
				\item Uno de los \textbf{hotspots de biodiversidad} más amenazados del planeta \cite{wwf2024atlantic}
			\end{itemize}
		\end{column}
		\begin{column}{0.5\textwidth}
			% TODO: Mapa del Bosque Atlántico historico vs actual
			\centering
			\fbox{\parbox{0.9\textwidth}{\centering\vspace{2cm}
					[Mapa Bosque Atlántico\\histórico vs. actual]
					\vspace{2cm}}}
		\end{column}
	\end{columns}
\end{frame}

\begin{frame}{La Selva Misionera}
	\begin{columns}
		\begin{column}{0.5\textwidth}
			\textbf{Remanente continuo mas extenso} del Bosque Atlantico en el Cono Sur

			\vspace{1em}
			\begin{itemize}
				\item 1.1 millones de hectareas protegidas
				\item \textbf{Corredor Verde de Misiones}
				\item Mas del \textbf{50\% de la biodiversidad} de Argentina \cite{dibitetti2003yaguarete}
				\item En menos del 0.5\% del territorio nacional
			\end{itemize}
		\end{column}
		\begin{column}{0.5\textwidth}
			% TODO: Mapa de la Selva Misionera y Corredor Verde
			\centering
			\fbox{\parbox{0.9\textwidth}{\centering\vspace{2cm}
					[Mapa Selva Misionera\\Corredor Verde]
					\vspace{2cm}}}
		\end{column}
	\end{columns}
\end{frame}

\begin{frame}{Biodiversidad Excepcional}
	\begin{columns}
		\begin{column}{0.55\textwidth}
			\textbf{Concentracion de especies:}
			\begin{itemize}
				\item $\sim$3,000 especies de plantas vasculares
				\item 554 especies de aves
				\item 120 especies de mamiferos
				\item 79 reptiles y 55 anfibios
			\end{itemize}

			\vspace{0.5em}
			\textbf{Especies emblematicas:}
			\begin{itemize}
				\item Yaguarete (\textit{Panthera onca})
				\item Tapir (\textit{Tapirus terrestris})
				\item Aguila harpia (\textit{Harpia harpyja})
				\item Yacutinga (\textit{Aburria jacutinga})
			\end{itemize}
		\end{column}
		\begin{column}{0.45\textwidth}
			% TODO: Collage de fauna emblematica
			\centering
			\fbox{\parbox{0.9\textwidth}{\centering\vspace{2.5cm}
					[Collage fauna emblematica]
					\vspace{2.5cm}}}
		\end{column}
	\end{columns}
\end{frame}

\begin{frame}{El Yaguarete: Especie Bandera}
	\begin{columns}
		\begin{column}{0.5\textwidth}
			\begin{itemize}
				\item \textbf{Monumento Natural} provincial (1988) y nacional (2001)
				\item Censo 2024: $\sim$84 individuos en el Corredor Verde \cite{vidasilvestre2024censo}
				\item Menos de 250 adultos en toda Argentina
				\item \textbf{En peligro critico} de extincion
			\end{itemize}

			\vspace{0.5em}
			Su presencia es \textbf{indicador clave} del estado de salud del ecosistema
		\end{column}
		\begin{column}{0.5\textwidth}
			% TODO: Foto de yaguarete (camara trampa)
			\centering
			\fbox{\parbox{0.9\textwidth}{\centering\vspace{2cm}
					[Foto yaguarete\\camara trampa]
					\vspace{2cm}}}
		\end{column}
	\end{columns}
\end{frame}

\begin{frame}{Importancia Ecologica}
	\textbf{Servicios ecosistemicos:}
	\begin{itemize}
		\item Regulacion del ciclo hidrologico
		\item Secuestro de carbono
		\item Proteccion contra la erosion del suelo
		\item Regulacion climatica regional
	\end{itemize}

	\vspace{1em}
	\textbf{Marco legal:} Ley 26.331 de Proteccion de Bosques Nativos \cite{ley26331bosques}
	\begin{description}
		\item[Categoria I (Rojo):] Muy alto valor - no se transforma
		\item[Categoria II (Amarillo):] Mediano valor - uso sostenible
		\item[Categoria III (Verde):] Bajo valor - puede transformarse
	\end{description}
\end{frame}

\begin{frame}{Amenazas a la Conservación}
	% TODO: Imagen ilustrativa de las amenazas
	\centering
	\fbox{\parbox{0.6\textwidth}{\centering\vspace{1.5cm}
			[Imagen amenazas a la conservacion]
			\vspace{1.5cm}}}

	\begin{itemize}
		\item Deforestacion y fragmentacion del habitat
		\item Caza furtiva y trafico de fauna
		\item Intrusion en areas protegidas
		\item Recursos limitados para vigilancia
	\end{itemize}
\end{frame}

\begin{frame}{Deforestacion}
	\begin{columns}
		\begin{column}{0.5\textwidth}
			\textbf{1990-2020:} $\sim$130,000 hectareas perdidas solo en el Corredor Verde \cite{fauba2024corredorverde}

			\vspace{0.5em}
			\begin{itemize}
				\item 77\% en parcelas $<$50 ha
				\item Ocupaciones para cultivos de subsistencia
				\item Tala ilegal de madera noble
			\end{itemize}

			\vspace{0.5em}
			\textbf{2025:} Reduccion del 18\% respecto al promedio historico \cite{ecologiamisiones2025deforestacion}
		\end{column}
		\begin{column}{0.5\textwidth}
			% TODO: Grafico/mapa de deforestacion
			\centering
			\fbox{\parbox{0.9\textwidth}{\centering\vspace{2cm}
					[Mapa deforestacion\\o grafico temporal]
					\vspace{2cm}}}
		\end{column}
	\end{columns}
\end{frame}

\begin{frame}{Fragmentacion del Habitat}
	\begin{alertblock}{Consecuencias}
		\begin{itemize}
			\item Poblaciones aisladas geneticamente
			\item Desplazamiento por areas no protegidas
			\item Conflictos con actividades humanas
			\item Atropellamientos en rutas
		\end{itemize}
	\end{alertblock}

	% TODO: Diagrama de fragmentacion
	\centering
	\fbox{\parbox{0.7\textwidth}{\centering\vspace{1.5cm}
			[Diagrama fragmentacion del habitat]
			\vspace{1.5cm}}}
\end{frame}

\begin{frame}{Caza Furtiva}
	\begin{columns}
		\begin{column}{0.55\textwidth}
			\textbf{Dos dimensiones:}
			\begin{enumerate}
				\item \textbf{Cultural/subsistencia:} Residentes locales
				\item \textbf{Economica/organizada:} Trafico de fauna
			\end{enumerate}

			\vspace{0.5em}
			\textbf{Zonas criticas:}
			\begin{itemize}
				\item Frontera con Brasil
				\item Reserva de Biosfera Yaboti
				\item Parques provinciales Pinalito, Urugua-i
			\end{itemize}

			\vspace{0.5em}
			\textbf{Especies afectadas:} Tapir, paca, corzuelas, tucanes, loros
		\end{column}
		\begin{column}{0.45\textwidth}
			% TODO: Imagen relacionada a caza furtiva
			\centering
			\fbox{\parbox{0.9\textwidth}{\centering\vspace{2cm}
					[Imagen problematica\\caza furtiva]
					\vspace{2cm}}}
		\end{column}
	\end{columns}
\end{frame}

\begin{frame}{Intrusion en Areas Protegidas}
	\textbf{Actividades ilicitas frecuentes:}
	\begin{itemize}
		\item Pesca ilegal en cursos de agua
		\item Desmonte encubierto para expansion de cultivos
		\item Campamentos de caza con infraestructura permanente
		\item Extraccion de recursos naturales
	\end{itemize}

	\vspace{1em}
	\begin{alertblock}{Problema critico}
		Sin un sistema de \textbf{alerta temprana}, las intrusiones se descubren \textbf{a posteriori} durante patrullajes de rutina, cuando el dano ya fue perpetrado.
	\end{alertblock}
\end{frame}

\begin{frame}{Desafios de la Vigilancia}
	\textbf{780,000 hectareas} distribuidas en \textbf{106+ areas protegidas}

	\vspace{0.5em}
	\begin{description}
		\item[Extension:] Terreno accidentado, vegetacion densa, acceso limitado
		\item[Comunicaciones:] Sin cobertura celular en zonas interiores
		\item[Latencia:] Semanas/meses entre captura de evento y descubrimiento
		\item[Volumen:] Miles de imagenes requieren clasificacion manual
	\end{description}

	% TODO: Foto de terreno/vegetacion densa
	\centering
	\fbox{\parbox{0.5\textwidth}{\centering\vspace{1.2cm}
			[Foto vegetacion densa]
			\vspace{1.2cm}}}
\end{frame}

% ==============================================================================
% Sección 2: Sistemas Actuales y Limitaciones
% ==============================================================================
\section{Sistemas Actuales}

\begin{frame}{Camaras Trampa Tradicionales \cite{steenweg2017scaling}}
	\begin{columns}
		\begin{column}{0.5\textwidth}
			\textbf{Componentes:}
			\begin{itemize}
				\item Sensor PIR (movimiento)
				\item Camara digital
				\item Almacenamiento SD
				\item Iluminacion IR (nocturna)
				\item Baterias AA
			\end{itemize}

			\textbf{Ventajas:}
			\begin{itemize}
				\item Alta autonomia
				\item Bajo costo inicial
				\item Robustez probada
			\end{itemize}
		\end{column}
		\begin{column}{0.5\textwidth}
			% TODO: Foto camara trampa tradicional
			\centering
			\fbox{\parbox{0.9\textwidth}{\centering\vspace{2cm}
					[Foto camara trampa tradicional]
					\vspace{2cm}}}
		\end{column}
	\end{columns}
\end{frame}

\begin{frame}{Limitaciones de Camaras Tradicionales}
	\begin{alertblock}{Brechas criticas}
		\begin{description}
			\item[Latencia:] Imagenes almacenadas localmente por semanas/meses
			\item[Sin alertas:] Informacion fluye solo hacia centros de analisis
			\item[Manual:] 80\%+ de imagenes son vacias o irrelevantes \cite{tabak2019machine}
			\item[Mantenimiento:] Visitas periodicas para baterias y tarjetas
		\end{description}
	\end{alertblock}

	\vspace{0.5em}
	\centering
	\textit{Para cuando se detecta una intrusion,\\los responsables ya estan lejos del area.}
\end{frame}

\begin{frame}{Soluciones Comerciales Celulares}
	\begin{columns}
		\begin{column}{0.55\textwidth}
			\textbf{Ejemplos:} Tactacam REVEAL \cite{tactacam2024reveal}, Spypoint Flex \cite{spypoint2024cellular}

			\vspace{0.5em}
			\textbf{Caracteristicas:}
			\begin{itemize}
				\item Envio via redes 3G/4G/LTE
				\item App movil propietaria
				\item Disparo rapido ($<$0.5s)
				\item Vision nocturna IR
			\end{itemize}
		\end{column}
		\begin{column}{0.45\textwidth}
			% TODO: Imagen camara celular comercial
			\centering
			\fbox{\parbox{0.9\textwidth}{\centering\vspace{2cm}
					[Camara celular comercial]
					\vspace{2cm}}}
		\end{column}
	\end{columns}
\end{frame}

\begin{frame}{Limitaciones de Soluciones Comerciales}
	\begin{block}{Dependencia de infraestructura}
		\begin{itemize}
			\item \textbf{Sin cobertura celular} en interior de reservas
			\item En selva densa, la senal es inexistente
		\end{itemize}
	\end{block}

	\begin{block}{Costos elevados}
		\begin{itemize}
			\item Alto costo de adquisicion por unidad
			\item Suscripciones mensuales: \$5-15 USD por camara
			\item Servicios en la nube propietarios
		\end{itemize}
	\end{block}

	\begin{block}{Escalabilidad limitada}
		\begin{itemize}
			\item Costo prohibitivo para despliegues masivos
			\item Especialmente en paises en vias de desarrollo
		\end{itemize}
	\end{block}
\end{frame}

\begin{frame}{Trabajos Relacionados}
	\begin{description}
		\item[Barrero \& Schmunck (UNaM, 2023) \cite{barrero2023microcamara}:] Microcamara de vigilancia para fauna - bases para soluciones de bajo costo

		\item[Whytock et al. (2023) \cite{whytock2023iridium}:] Camaras trampa con IA + alertas satelitales Iridium - costos operativos muy altos

		\item[AiCatcher (Mallya, 2019) \cite{mallya2019aicatcher}:] Raspberry Pi + LoRa + inferencia en borde - alto consumo energetico
	\end{description}

	\vspace{0.5em}
	\begin{alertblock}{Brecha identificada}
		Oportunidad para un sistema que combine \textbf{bajo costo}, \textbf{independencia de infraestructura celular}, y \textbf{alertas en tiempo real}.
	\end{alertblock}
\end{frame}

\begin{frame}{Analisis Comparativo}
	\begin{table}
		\scriptsize
		\centering
		\begin{tabular}{lcccc}
			\hline
			\textbf{Caracteristica} & \textbf{Tradicional} & \textbf{Celular} & \textbf{Satelital} & \textbf{LoRa} \\
			\hline
			Alertas tiempo real     & No                   & Si               & Si                 & Si            \\
			Infraestructura         & Ninguna              & Operador         & Satelite           & Gateway       \\
			Procesamiento IA        & Post-hoc             & No/Limitado      & Edge               & Edge          \\
			Costo adquisicion       & Bajo                 & Medio-Alto       & Muy alto           & Medio         \\
			Costo operativo         & Bajo                 & Alto             & Muy alto           & Bajo          \\
			Autonomia               & Muy alta             & Media            & Baja               & Baja          \\
			\hline
		\end{tabular}
	\end{table}

	\vspace{0.5em}
	\centering
	\textit{Las alertas inmediatas estan condicionadas por\\elevados costos o dependencia de terceros.}
\end{frame}

% ==============================================================================
% Sección 3: Motivación y Objetivos
% ==============================================================================
\section{Motivacion y Objetivos}

\begin{frame}{Motivacion del Proyecto}
	\textbf{Transformar el paradigma:}

	\begin{center}
		Monitoreo \textbf{pasivo} $\longrightarrow$ Vigilancia \textbf{activa e inteligente}
	\end{center}

	\vspace{1em}
	\textbf{Un sistema que:}
	\begin{itemize}
		\item No solo capture imagenes, sino que las \textbf{transmita en tiempo real}
		\item Las \textbf{analice automaticamente} mediante IA
		\item Genere \textbf{alertas inmediatas} ante eventos de interes
		\item Sea \textbf{accesible y de bajo costo}
		\item Use \textbf{hardware economico} y \textbf{software de codigo abierto}
	\end{itemize}
\end{frame}

\begin{frame}{Oportunidad Tecnologica}
	\textbf{Convergencia de tres desarrollos:}

	\vspace{0.5em}
	\begin{block}{1. Microcontroladores de bajo costo}
		ESP32: WiFi, Bluetooth, bajo consumo. Costo: $<$\$10 USD
	\end{block}

	\begin{block}{2. Redes mesh autoorganizadas}
		ESP-MESH: Extension de cobertura sin infraestructura celular
	\end{block}

	\begin{block}{3. IA para vision por computadora}
		SpeciesNet, MegaDetector: Modelos de codigo abierto para clasificacion automatica de fauna
	\end{block}
\end{frame}

\begin{frame}{Objetivo General}
	\begin{alertblock}{Objetivo}
		Demostrar la \textbf{viabilidad tecnica} de un sistema de monitoreo de \textbf{bajo costo} para areas protegidas, basado en:
		\begin{itemize}
			\item Nodos de camara con conectividad \textbf{mesh}
			\item Clasificacion automatica de imagenes mediante \textbf{IA}
		\end{itemize}
	\end{alertblock}

	\vspace{1em}
	\textbf{Reduccion del tiempo de procesamiento:}
	\begin{center}
		\textbf{Dias/semanas} $\longrightarrow$ \textbf{Minutos}
	\end{center}
\end{frame}

\begin{frame}{Objetivos Especificos}
	\begin{enumerate}
		\item Disenar e implementar un \textbf{nodo de camara autonomo} basado en ESP32

		\item Desarrollar una \textbf{red mesh autoorganizada} con ESP-MESH

		\item Implementar un \textbf{servidor de procesamiento} con IA

		\item Integrar \textbf{SpeciesNet/MegaDetector} para clasificacion

		\item \textbf{Validar el funcionamiento} en condiciones reales
	\end{enumerate}
\end{frame}

\begin{frame}{Alcance del Proyecto}
	\begin{columns}
		\begin{column}{0.5\textwidth}
			\textbf{Dentro del alcance:}
			\begin{itemize}
				\item Nodos ESP32-CAM + sensor PIR
				\item Red mesh funcional
				\item Servidor de recepcion y clasificacion
				\item Bot de Telegram para alertas
				\item Pruebas en laboratorio y campo
			\end{itemize}
		\end{column}
		\begin{column}{0.5\textwidth}
			\textbf{Fuera del alcance:}
			\begin{itemize}
				\item Despliegue en selva densa
				\item Carcasas con grado IP
				\item Entrenamiento de modelos especificos
				\item Certificacion comercial
			\end{itemize}
		\end{column}
	\end{columns}
\end{frame}

\begin{frame}{Limitaciones Conocidas}
	\begin{description}
		\item[Operacion diurna:] Lente con filtro IR, sin vision nocturna
		\item[Sensor PIR:] Optimizado para humanos, puede no detectar fauna pequena
		\item[Resolucion:] Limitada por tamano maximo de paquete mesh ($<$8KB)
		\item[Consumo:] WiFi consume mas que camaras trampa comerciales optimizadas
	\end{description}

	\vspace{0.5em}
	\begin{block}{Nota sobre ``tiempo real''}
		No es tiempo real deterministico (milisegundos), sino \textbf{operativo}: de dias/semanas a \textbf{minutos}.
	\end{block}
\end{frame}

% ==============================================================================
% Sección 4: Marco Teórico
% ==============================================================================
\section{Marco Teorico}

\begin{frame}{Funcionamiento de Camaras Trampa}
	\textbf{Ciclo de operacion:}
	\begin{enumerate}
		\item \textbf{Reposo:} Solo sensor PIR activo (bajo consumo)
		\item \textbf{Deteccion:} Cambio en radiacion IR
		\item \textbf{Activacion:} Despertar camara (trigger time: 0.1-2s)
		\item \textbf{Captura:} Foto/video + metadatos
		\item \textbf{Almacenamiento:} Compresion JPEG, grabacion en SD
		\item \textbf{Retorno al reposo}
	\end{enumerate}

	% TODO: Diagrama del ciclo
	\centering
	\fbox{\parbox{0.6\textwidth}{\centering\vspace{1cm}
			[Diagrama ciclo operacion]
			\vspace{1cm}}}
\end{frame}

\begin{frame}{Internet de las Cosas (IoT)}
	\textbf{Arquitectura de tres capas:}

	% TODO: Diagrama arquitectura IoT
	\centering
	\fbox{\parbox{0.7\textwidth}{\centering\vspace{2cm}
			[Diagrama arquitectura IoT - 3 capas]
			\vspace{2cm}}}

	\begin{enumerate}
		\item \textbf{Percepcion:} Sensores, camaras $\rightarrow$ ESP32-CAM + PIR
		\item \textbf{Red:} Transmision de datos $\rightarrow$ ESP-MESH
		\item \textbf{Aplicacion:} Procesamiento, alertas $\rightarrow$ Django + SpeciesNet
	\end{enumerate}
\end{frame}

\begin{frame}{Protocolos de Comunicacion}
	\begin{table}
		\small
		\centering
		\begin{tabular}{lccc}
			\hline
			\textbf{Protocolo} & \textbf{Alcance} & \textbf{Ancho banda} & \textbf{Consumo} \\
			\hline
			BLE                & $\sim$100m       & Bajo                 & Muy bajo         \\
			ZigBee             & $\sim$100m       & Bajo                 & Bajo             \\
			LoRa               & km               & Muy bajo             & Bajo             \\
			WiFi               & $\sim$100m       & \textbf{Alto}        & Alto             \\
			\hline
		\end{tabular}
	\end{table}

	\vspace{0.5em}
	\textbf{LoRa:} Excelente alcance, pero ancho de banda \textbf{insuficiente} para imagenes

	\vspace{0.5em}
	\textbf{WiFi Mesh:} Permite transmision de imagenes + extension de cobertura
\end{frame}

\begin{frame}{Redes Mesh - ESP-MESH \cite{espmesh2023}}
	\begin{columns}
		\begin{column}{0.5\textwidth}
			\textbf{Caracteristicas:}
			\begin{itemize}
				\item \textbf{Autoorganizacion:} Seleccion automatica de padre optimo
				\item \textbf{Autocuracion:} Reconexion ante fallos
				\item Topologia de arbol
				\item Hasta 6+ niveles de profundidad
				\item Cientos de nodos
			\end{itemize}
		\end{column}
		\begin{column}{0.5\textwidth}
			% TODO: Diagrama topologia mesh
			\centering
			\fbox{\parbox{0.9\textwidth}{\centering\vspace{2cm}
					[Diagrama topologia mesh]
					\vspace{2cm}}}
		\end{column}
	\end{columns}
\end{frame}

\begin{frame}{Deteccion de Objetos con YOLO \cite{redmon2016yolo}}
	\begin{columns}
		\begin{column}{0.5\textwidth}
			\textbf{You Only Look Once}
			\begin{itemize}
				\item Deteccion en una sola pasada
				\item Bounding boxes + probabilidades
				\item Balance velocidad/precision
				\item YOLOv5: PyTorch, multiples variantes \cite{jocher2020yolov5}
			\end{itemize}
		\end{column}
		\begin{column}{0.5\textwidth}
			% TODO: Ejemplo deteccion YOLO
			\centering
			\fbox{\parbox{0.9\textwidth}{\centering\vspace{2cm}
					[Ejemplo deteccion YOLO\\con bounding boxes]
					\vspace{2cm}}}
		\end{column}
	\end{columns}
\end{frame}

\begin{frame}{MegaDetector \cite{beery2019megadetector}}
	\textbf{Microsoft AI for Earth}

	\vspace{0.5em}
	Modelo especializado para camaras trampa que detecta:
	\begin{itemize}
		\item \textbf{Animales} (sin distincion de especie)
		\item \textbf{Personas}
		\item \textbf{Vehiculos}
	\end{itemize}

	\vspace{0.5em}
	\textbf{Ventajas:}
	\begin{itemize}
		\item Altamente robusto y generalizable
		\item Adoptado por 60+ organizaciones de conservacion \cite{velez2022platforms}
		\item Reduce hasta 90\% el tiempo de procesamiento
	\end{itemize}
\end{frame}

\begin{frame}{SpeciesNet - Google \cite{gadot2024crop}}
	\textbf{Pipeline de dos fases:}

	\begin{enumerate}
		\item \textbf{Deteccion:} Identifica regiones de interes (animales, personas, vehiculos)

		\item \textbf{Clasificacion taxonomica:} Para animales detectados
		      \begin{itemize}
			      \item Familia
			      \item Genero
			      \item Especie
		      \end{itemize}
	\end{enumerate}

	\vspace{0.5em}
	\textbf{Entrenado con millones de imagenes} de camaras trampa a nivel global

	% TODO: Ejemplo de clasificacion
	\centering
	\fbox{\parbox{0.5\textwidth}{\centering\vspace{1cm}
			[Ejemplo clasificacion taxonomica]
			\vspace{1cm}}}
\end{frame}

% ==============================================================================
% Sección 5: Arquitectura del Sistema
% ==============================================================================
\section{Arquitectura del Sistema}

\begin{frame}{Vision General}
	% TODO: Diagrama arquitectura general
	\centering
	\fbox{\parbox{0.9\textwidth}{\centering\vspace{3cm}
			[Diagrama arquitectura general del sistema]\\
			Nodos camara $\rightarrow$ Red Mesh $\rightarrow$ Nodo Raiz $\rightarrow$ Servidor $\rightarrow$ Telegram
			\vspace{3cm}}}
\end{frame}

\begin{frame}{Flujo de Datos}
	\begin{enumerate}
		\item Sensor PIR detecta movimiento $\rightarrow$ interrupcion
		\item Nodo captura imagen JPEG ($<$8KB)
		\item Transmision via red mesh hasta nodo raiz
		\item Nodo raiz envia HTTP POST al servidor
		\item Servidor procesa con SpeciesNet
		\item Si hay deteccion: alerta via Telegram
		\item Almacenamiento en base de datos
	\end{enumerate}
\end{frame}

\begin{frame}{Nodo de Camara - Hardware}
	\begin{columns}
		\begin{column}{0.5\textwidth}
			\textbf{Componentes:}
			\begin{itemize}
				\item ESP32-CAM (AI-Thinker)
				\item Camara OV2640 (2MP)
				\item Sensor PIR HC-SR501
				\item Buck converter LM2596
				\item Baterias 18650 (2S, 7.4V)
			\end{itemize}

			\vspace{0.5em}
			\textbf{Resolucion:} 640x480 (VGA)
		\end{column}
		\begin{column}{0.5\textwidth}
			% TODO: Foto del nodo ensamblado
			\centering
			\fbox{\parbox{0.9\textwidth}{\centering\vspace{2cm}
					[Foto nodo camara ensamblado]
					\vspace{2cm}}}
		\end{column}
	\end{columns}
\end{frame}

\begin{frame}{Nodo de Camara - Firmware}
	\textbf{Desarrollado con ESP-IDF + ESP-MDF}

	\vspace{0.5em}
	\textbf{Funcionalidades:}
	\begin{itemize}
		\item Inicializacion de camara OV2640
		\item Interrupcion GPIO para sensor PIR
		\item Conexion a red ESP-MESH como nodo hijo
		\item Captura y compresion JPEG
		\item Almacenamiento en microSD (respaldo)
		\item Transmision via Mwifi
		\item Modo power-save para ahorro de energia
	\end{itemize}
\end{frame}

\begin{frame}{Carcasa Impresa en 3D}
	\begin{columns}
		\begin{column}{0.5\textwidth}
			\textbf{Diseno basado en modelo CC 4.0}

			\vspace{0.5em}
			\textbf{Caracteristicas:}
			\begin{itemize}
				\item Material: PLA
				\item Apertura para lente
				\item Domo para sensor PIR
				\item Espacio para buck converter
				\item Acceso a tarjeta SD
				\item Puntos de montaje
			\end{itemize}

			\vspace{0.5em}
			\textit{Disponible en Printables}
		\end{column}
		\begin{column}{0.5\textwidth}
			% TODO: Render/foto carcasa
			\centering
			\fbox{\parbox{0.9\textwidth}{\centering\vspace{2cm}
					[Render/foto carcasa 3D]
					\vspace{2cm}}}
		\end{column}
	\end{columns}
\end{frame}

\begin{frame}{Nodo Raiz}
	\begin{columns}
		\begin{column}{0.55\textwidth}
			\textbf{Hardware:} ESP32 DevKit V1 (sin camara)

			\vspace{0.5em}
			\textbf{Rol:} Puerta de enlace Mesh $\leftrightarrow$ Internet

			\vspace{0.5em}
			\textbf{Funciones:}
			\begin{itemize}
				\item Raiz de la red ESP-MESH
				\item Recepcion de imagenes de nodos hijos
				\item Conexion WiFi a router
				\item Cliente HTTP para envio al servidor
				\item Sincronizacion de tiempo
			\end{itemize}
		\end{column}
		\begin{column}{0.45\textwidth}
			% TODO: Diagrama flujo nodo raiz
			\centering
			\fbox{\parbox{0.9\textwidth}{\centering\vspace{2cm}
					[Diagrama nodo raiz]
					\vspace{2cm}}}
		\end{column}
	\end{columns}
\end{frame}

\begin{frame}{Servidor de Aplicacion}
	\textbf{Arquitectura containerizada con Docker Compose:}

	\vspace{0.5em}
	\begin{description}
		\item[Django:] Backend web, recepcion de imagenes
		\item[SpeciesNet:] Servicio de inferencia (LitServe)
		\item[PostgreSQL:] Base de datos
		\item[Telegram Bot:] Notificaciones
	\end{description}

	% TODO: Diagrama Docker Compose
	\centering
	\fbox{\parbox{0.6\textwidth}{\centering\vspace{1.5cm}
			[Diagrama Docker Compose]
			\vspace{1.5cm}}}
\end{frame}

\begin{frame}{Sistema de Alertas}
	\begin{columns}
		\begin{column}{0.5\textwidth}
			\textbf{Bot de Telegram}

			\vspace{0.5em}
			Cada notificacion incluye:
			\begin{itemize}
				\item Imagen original
				\item Imagen anotada (bounding boxes)
				\item Cantidad y confianza de detecciones
				\item Clasificaciones taxonomicas
			\end{itemize}

			\vspace{0.5em}
			\textbf{Comandos:}
			\begin{itemize}
				\item \texttt{/start} - Registro
				\item \texttt{/last} - Ultima imagen
			\end{itemize}
		\end{column}
		\begin{column}{0.5\textwidth}
			% TODO: Screenshot Telegram
			\centering
			\fbox{\parbox{0.9\textwidth}{\centering\vspace{2.5cm}
					[Screenshot notificacion Telegram]
					\vspace{2.5cm}}}
		\end{column}
	\end{columns}
\end{frame}

% ==============================================================================
% Sección 6: Metodología
% ==============================================================================
\section{Metodologia}

\begin{frame}{Enfoque de Desarrollo}
	\textbf{Iterativo e incremental}

	\vspace{0.5em}
	\begin{itemize}
		\item Cada componente desarrollado, probado y refinado individualmente
		\item Identificacion temprana de problemas
		\item Metodologia agil \textbf{Kanban}
	\end{itemize}

	\vspace{0.5em}
	\textbf{Control de versiones:} GitHub (4 repositorios)

	\vspace{0.5em}
	\textbf{Despliegue continuo:} GitHub Actions
	\begin{itemize}
		\item Build automatico de imagenes Docker
		\item Publicacion en GitHub Container Registry
	\end{itemize}
\end{frame}

\begin{frame}{Etapas del Desarrollo}
	% TODO: Diagrama Gantt simplificado
	\centering
	\fbox{\parbox{0.8\textwidth}{\centering\vspace{2cm}
			[Diagrama Gantt - etapas del desarrollo]
			\vspace{2cm}}}

	\begin{enumerate}
		\item Investigacion y diseno
		\item Desarrollo del firmware
		\item Desarrollo del servidor
		\item Diseno y fabricacion de hardware
		\item Integracion
		\item Pruebas y validacion
	\end{enumerate}
\end{frame}

\begin{frame}{Herramientas Utilizadas}
	\begin{columns}
		\begin{column}{0.5\textwidth}
			\textbf{Desarrollo:}
			\begin{itemize}
				\item VS Code
				\item ESP-IDF + ESP-MDF
				\item Django
				\item python-telegram-bot
				\item Docker / Docker Compose
			\end{itemize}
		\end{column}
		\begin{column}{0.5\textwidth}
			\textbf{Diseno 3D:}
			\begin{itemize}
				\item Autodesk Fusion 360
				\item PrusaSlicer
				\item Impresora Creality Ender 3
			\end{itemize}
		\end{column}
	\end{columns}
\end{frame}

\begin{frame}{Desafios Enfrentados}
	\begin{alertblock}{Desafio 1: Limite de tamano de paquete}
		ESP-MDF limita paquetes a 8KB $\rightarrow$ ajustar compresion JPEG
	\end{alertblock}

	\begin{alertblock}{Desafio 2: Consumo energetico}
		WiFi consume significativamente $\rightarrow$ modo power-save de Mwifi
	\end{alertblock}

	\begin{alertblock}{Desafio 3: Estabilidad de red mesh}
		Reconexiones frecuentes $\rightarrow$ tuning de parametros de conexion
	\end{alertblock}

	\begin{alertblock}{Desafio 4: Tiempo de inferencia}
		SpeciesNet lento en CPU $\rightarrow$ soporte para GPU (nvidia-container-toolkit)
	\end{alertblock}
\end{frame}

% ==============================================================================
% Sección 7: Pruebas y Resultados
% ==============================================================================
\section{Pruebas y Resultados}

\begin{frame}{Configuracion de Pruebas}
	\begin{columns}
		\begin{column}{0.5\textwidth}
			\textbf{Hardware:}
			\begin{itemize}
				\item 3 nodos de camara
				\item 1 nodo raiz
				\item Baterias 18650 (2S)
				\item Router WiFi domestico
			\end{itemize}
		\end{column}
		\begin{column}{0.5\textwidth}
			\textbf{Software:}
			\begin{itemize}
				\item ESP-IDF v5.x
				\item Docker containers
				\item SpeciesNet + LitServe
				\item PostgreSQL
				\item Tunel VPN (Pangolin)
			\end{itemize}
		\end{column}
	\end{columns}

	% TODO: Foto setup
	\centering
	\fbox{\parbox{0.5\textwidth}{\centering\vspace{1cm}
			[Foto setup de pruebas]
			\vspace{1cm}}}
\end{frame}

\begin{frame}{Pruebas de Laboratorio}
	\textbf{Validaciones realizadas:}

	\begin{itemize}
		\item Captura de imagen: inicializacion, calidad, tamano
		\item Sensor PIR: distancia, tiempo de respuesta, falsas activaciones
		\item Transmision mesh: conexion, latencia, perdida de paquetes
		\item Clasificacion: deteccion de animales/personas/vehiculos
		\item Alertas: recepcion en Telegram, contenido correcto
	\end{itemize}
\end{frame}

\begin{frame}{Resultados - Captura de Imagen}
	% TODO: Ejemplo imagen capturada
	\centering
	\fbox{\parbox{0.6\textwidth}{\centering\vspace{2cm}
			[Ejemplo imagen capturada 640x480]
			\vspace{2cm}}}

	\begin{itemize}
		\item Resolucion: 640x480 (VGA)
		\item Formato: JPEG comprimido
		\item Tamano tipico: $<$8KB
	\end{itemize}
\end{frame}

\begin{frame}{Resultados - Red Mesh}
	\begin{table}
		\centering
		\begin{tabular}{lc}
			\hline
			\textbf{Metrica}               & \textbf{Valor} \\
			\hline
			Latencia promedio              & [X] s          \\
			Distancia max (linea de vista) & [X] m          \\
			Distancia max (con obstaculos) & [X] m          \\
			Tasa perdida de paquetes       & $<$5\%         \\
			Tiempo de reconexion           & [X] s          \\
			\hline
		\end{tabular}
	\end{table}
\end{frame}

\begin{frame}{Resultados - Deteccion}
	\begin{columns}
		\begin{column}{0.5\textwidth}
			\begin{table}
				\centering
				\begin{tabular}{lc}
					\hline
					\textbf{Categoria} & \textbf{Precision} \\
					\hline
					Animales           & [X]\%              \\
					Personas           & [X]\%              \\
					Vehiculos          & [X]\%              \\
					\hline
				\end{tabular}
			\end{table}

			\vspace{0.5em}
			\textbf{Tiempo de inferencia:}
			\begin{itemize}
				\item CPU: 1-5 s/imagen
				\item GPU: 0.5-1 s/imagen
			\end{itemize}
		\end{column}
		\begin{column}{0.5\textwidth}
			% TODO: Ejemplo deteccion exitosa
			\centering
			\fbox{\parbox{0.9\textwidth}{\centering\vspace{2cm}
					[Ejemplo deteccion exitosa]
					\vspace{2cm}}}
		\end{column}
	\end{columns}
\end{frame}

\begin{frame}{Resultados - Consumo Energetico}
	\begin{table}
		\centering
		\begin{tabular}{lc}
			\hline
			\textbf{Estado}    & \textbf{Consumo (mA)} \\
			\hline
			Modo power-save    & [X]                   \\
			Captura de imagen  & [X]                   \\
			Transmision WiFi   & [X]                   \\
			Promedio ponderado & [X]                   \\
			\hline
		\end{tabular}
	\end{table}

	\vspace{0.5em}
	\textbf{Autonomia estimada:} [X] horas con baterias 18650 (2S, [X]mAh)
\end{frame}

\begin{frame}{Pruebas de Campo}
	\begin{columns}
		\begin{column}{0.5\textwidth}
			\textbf{Condiciones:}
			\begin{itemize}
				\item Duracion: [X] horas
				\item 3 nodos desplegados
				\item Distancia entre nodos: [X]m
				\item Condiciones climaticas favorables
			\end{itemize}

			\vspace{0.5em}
			\textbf{Resultados:}
			\begin{itemize}
				\item Imagenes capturadas: [X]
				\item Detecciones exitosas: [X]
				\item Falsas activaciones: [X]
			\end{itemize}
		\end{column}
		\begin{column}{0.5\textwidth}
			% TODO: Foto despliegue campo
			\centering
			\fbox{\parbox{0.9\textwidth}{\centering\vspace{2cm}
					[Foto nodo desplegado en campo]
					\vspace{2cm}}}
		\end{column}
	\end{columns}
\end{frame}

% ==============================================================================
% Sección 8: Conclusiones
% ==============================================================================
\section{Conclusiones}

\begin{frame}{Logros Alcanzados}
	\begin{itemize}
		\item[$\checkmark$] Sistema completo de monitoreo con \textbf{hardware de bajo costo}
		\item[$\checkmark$] Reduccion de tiempo de procesamiento: \textbf{dias $\rightarrow$ minutos}
		\item[$\checkmark$] Red mesh funcional con \textbf{ESP-MESH}
		\item[$\checkmark$] Integracion exitosa de \textbf{SpeciesNet/MegaDetector}
		\item[$\checkmark$] Sistema de alertas en tiempo real via \textbf{Telegram}
		\item[$\checkmark$] Carcasa imprimible en 3D
		\item[$\checkmark$] \textbf{Codigo abierto} disponible en GitHub
	\end{itemize}
\end{frame}

\begin{frame}{Aportes del Trabajo}
	\begin{block}{Arquitectura integrada}
		Captura distribuida + transmision mesh + procesamiento IA centralizado
	\end{block}

	\begin{block}{Prototipo de bajo costo}
		Componentes comerciales economicos + disenos 3D listos para fabricacion
	\end{block}

	\begin{block}{Integracion de SpeciesNet}
		Pipeline de clasificacion taxonomica en tiempo operativo
	\end{block}

	\begin{block}{Codigo abierto}
		4 repositorios publicos en GitHub
	\end{block}
\end{frame}

\begin{frame}{Comparacion con Objetivos}
	\begin{table}
		\centering
		\begin{tabular}{lcc}
			\hline
			\textbf{Metrica}       & \textbf{Objetivo} & \textbf{Resultado} \\
			\hline
			Latencia de respuesta  & $<$X min          & [X] min            \\
			Precision de deteccion & $>$X\%            & [X]\%              \\
			Autonomia              & $>$X horas        & [X] horas          \\
			Cobertura              & $>$X m            & [X] m              \\
			\hline
		\end{tabular}
	\end{table}
\end{frame}

\begin{frame}{Trabajos Futuros}
	\begin{description}
		\item[Vision nocturna:] Camaras con capacidad IR
		\item[Conectividad largo alcance:] LoRa, 4G/LTE, satelital
		\item[Optimizacion energetica:] Paneles solares, nodo repetidor dedicado
		\item[Procesamiento en borde:] TinyML en nodos de camara
		\item[Modelos especificos:] Entrenamiento para fauna regional
		\item[Interfaz web:] Mapas, gestion de despliegues, reportes
	\end{description}
\end{frame}

\begin{frame}{Recomendaciones}
	\begin{enumerate}
		\item \textbf{Ubicaciones:} Considerar distancia entre nodos y obstaculos
		\item \textbf{Proteccion:} Usar PETG/ASA para despliegues permanentes
		\item \textbf{Baterias:} Dimensionar segun frecuencia de activaciones
		\item \textbf{Sensor PIR:} Evaluar alternativas para fauna pequena
		\item \textbf{Monitoreo:} Implementar seguimiento de estado de nodos
	\end{enumerate}
\end{frame}

\begin{frame}{Repositorios}
	\small
	\begin{description}
		\item[mesh-node:] \url{github.com/fabcontigiani/mesh-node-capstone-project}
		\item[root-node:] \url{github.com/fabcontigiani/root-node-capstone-project}
		\item[server:] \url{github.com/fabcontigiani/server-capstone-project}
		\item[wildlife-detection:] \url{github.com/fabcontigiani/wildlife-detection-capstone-project}
	\end{description}

	\vspace{1em}
	\textbf{Modelo 3D carcasa:} Disponible en Printables
\end{frame}

% ==============================================================================
% Cierre
% ==============================================================================

\begin{frame}{Agradecimientos}
	\begin{itemize}
		\item A nuestras familias por el apoyo incondicional
		\item Al Dr. Ing. Sergio Moya por su guía y dedicación como tutor
		\item A la Facultad de Ingeniería de la UNaM
		\item A los docentes que contribuyeron a nuestra formación
		\item A nuestros compañeros de carrera
	\end{itemize}

	\vspace{1em}
	\centering
	\textit{¡Gracias a todos!}
\end{frame}

\begin{frame}[standout]
	¿Preguntas?
\end{frame}

\begin{frame}[allowframebreaks]{Referencias}
	\printbibliography[heading=none]
\end{frame}

% ==============================================================================
% Slides de Backup
% ==============================================================================
\appendix

\begin{frame}{Esquemático del Hardware}
	% TODO: Esquematico KiCad
	\centering
	\fbox{\parbox{0.8\textwidth}{\centering\vspace{3cm}
			[Esquematico KiCad del nodo]
			\vspace{3cm}}}
\end{frame}

\begin{frame}{Planos de la Carcasa}
	% TODO: Planos 2D
	\centering
	\fbox{\parbox{0.8\textwidth}{\centering\vspace{3cm}
			[Planos 2D de la carcasa]
			\vspace{3cm}}}
\end{frame}

\begin{frame}{Análisis Económico}
	\begin{columns}
		\begin{column}{0.5\textwidth}
			\textbf{Inversión inicial:} \$20,530 USD

			\vspace{0.5em}
			\textbf{Modelo de negocio:}
			\begin{itemize}
				\item Cobro por instalación
				\item Suscripción mensual (\$50 USD)
			\end{itemize}

			\vspace{0.5em}
			\textbf{Mercado objetivo:}
			\begin{itemize}
				\item 116 reservas naturales
				\item 10,800+ EAPs con bosques
			\end{itemize}
		\end{column}
		\begin{column}{0.5\textwidth}
			\textbf{Indicadores:}
			\begin{itemize}
				\item VAN: \$7,517.55 USD
				\item TIR: 25\%
				\item TREMA: 15\%
				\item Recupero: Ano 4
			\end{itemize}

			\vspace{0.5em}
			\textbf{Conclusion:}\\
			Proyecto \textbf{economicamente viable}
		\end{column}
	\end{columns}
\end{frame}

\begin{frame}{Sensibilidad Económica}
	% TODO: Graficos de sensibilidad
	\centering
	\fbox{\parbox{0.8\textwidth}{\centering\vspace{2.5cm}
			[Graficos de análisis de sensibilidad]
			\vspace{2.5cm}}}

	\begin{itemize}
		\item Punto de equilibrio: 24 suscripciones/ano
		\item Precio minimo viable: \$40 USD/mes
	\end{itemize}
\end{frame}

\end{document}
