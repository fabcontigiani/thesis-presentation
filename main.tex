\documentclass[10pt]{beamer}
\usepackage[spanish]{babel}
\usepackage{csquotes}
\usepackage[T1]{fontenc}

% \setbeameroption{hide notes} % Only slides
%\setbeameroption{show only notes} % Only notes
\setbeameroption{show notes on second screen=bottom} % Both

\usepackage{graphicx}
\usepackage{colortbl}

\usetheme{moloch}
\molochset{block=fill}

\setbeamertemplate{page number in head/foot}[appendixframenumber]
\setbeamertemplate{section in toc}[sections numbered]

\usepackage{plex-serif}
\usepackage{plex-sans}
\usepackage{plex-mono}

% Unidades
\usepackage[locale=DE]{siunitx}

% Bibliografia
\usepackage[style=apa, backend=biber]{biblatex}
\renewrobustcmd*{\mkbibfootnote}[1]{\footnote[frame]{#1}}
\setbeamertemplate{footnote}{% Footnote spacing fix for double digits
  \hspace{2pt}%
  \makebox[1em][l]{\insertfootnotemark}%
  \insertfootnotetext\par%
}
\addbibresource{src/bachelor-thesis/referencias.bib}

% Links
\usepackage{hyperref}
\hypersetup{colorlinks=true,linkcolor=black,citecolor=.,urlcolor=blue}

\title{Sistema de Monitoreo y Alerta Temprana basado en IA para Áreas Protegidas}
\subtitle{Proyecto Final Integrador - Ingeniería en Computación}
\author{Fabrizio Contigiani - Gabriel Da Silva}
\institute{Tutor: Dr. Ing. Sergio Moya} % institute already in titlegraphic
\date{\today}
\titlegraphic{\includegraphics[width=\textwidth]{img/logos.png}}

\begin{document}

\maketitle

\begin{frame}
	\frametitle{Índice}
	\tableofcontents[hideallsubsections]
	\note{
		Presentar brevemente la estructura de la exposición. Mencionar que se
		cubrirán 8 secciones principales.
	}
\end{frame}

% ==============================================================================
% Sección 1: Contexto Regional
% ==============================================================================
\section{Contexto Regional}

\begin{frame}{El Bosque Atlántico}
	\begin{columns}
		\begin{column}{0.5\textwidth}
			\begin{itemize}
				\item Originalmente: \num{1.3} millones de \unit{\km\squared} \footcite{ribeiro2009atlantic}
				\item Brasil (\qty{92}{\percent}), Paraguay (\qty{6}{\percent}), Argentina (\qty{2}{\percent})
				\item Hoy: solo \qtyrange{12}{17}{\percent} de su extensión original
				\item Uno de los \textbf{hotspots de biodiversidad} más amenazados del planeta \footcite{wwf2024atlantic}
			\end{itemize}
		\end{column}
		\begin{column}{0.5\textwidth}
			% TODO: Mapa del Bosque Atlántico historico vs actual
			\centering
			\fbox{\parbox{0.9\textwidth}{\centering\vspace{2cm}
					[Mapa Bosque Atlántico\\histórico vs. actual]
					\vspace{2cm}}}
		\end{column}
	\end{columns}
	\note{
		El Bosque Atlántico es uno de los ecosistemas más biodiversos del mundo.
		Originalmente, cubría \num{1.3} millones de \unit{\km\squared}, pero hoy queda menos del \qty{17}{\percent}.
		Es considerado un hotspot de biodiversidad por su alta concentración de
		especies endémicas y el nivel de amenaza que enfrenta.
	}
\end{frame}

\begin{frame}{La Selva Misionera}
	\begin{columns}
		\begin{column}{0.5\textwidth}
			\textbf{Remanente continuo mas extenso} del Bosque Atlantico en el Cono Sur

			\vspace{1em}
			\begin{itemize}
				\item \num{1.1} millones de hectareas protegidas
				\item \textbf{Corredor Verde de Misiones}
				\item Mas del \textbf{\qty{50}{\percent} de la biodiversidad} de Argentina \footcite{dibitetti2003yaguarete}
				\item En menos del \qty{0.5}{\percent} del territorio nacional
			\end{itemize}
		\end{column}
		\begin{column}{0.5\textwidth}
			% TODO: Mapa de la Selva Misionera y Corredor Verde
			\centering
			\fbox{\parbox{0.9\textwidth}{\centering\vspace{2cm}
					[Mapa Selva Misionera\\Corredor Verde]
					\vspace{2cm}}}
		\end{column}
	\end{columns}
	\note{
		La Selva Misionera es el fragmento más grande y mejor conservado del
		Bosque Atlántico en Argentina. El Corredor Verde conecta áreas protegidas
		permitiendo el desplazamiento de fauna. Aquí se concentra más de la mitad
		de la biodiversidad del país en menos del \qty{0.5}{\percent} del territorio.
	}
\end{frame}

\begin{frame}{Biodiversidad Excepcional}
	\begin{columns}
		\begin{column}{0.55\textwidth}
			\textbf{Concentracion de especies:}
			\begin{itemize}
				\item $\sim$\num{3000} especies de plantas vasculares
				\item \num{554} especies de aves
				\item \num{120} especies de mamiferos
				\item \num{79} reptiles y \num{55} anfibios
			\end{itemize}

			\vspace{0.5em}
			\textbf{Especies emblematicas:}
			\begin{itemize}
				\item Yaguarete (\textit{Panthera onca})
				\item Tapir (\textit{Tapirus terrestris})
				\item Aguila harpia (\textit{Harpia harpyja})
				\item Yacutinga (\textit{Aburria jacutinga})
			\end{itemize}
		\end{column}
		\begin{column}{0.45\textwidth}
			% TODO: Collage de fauna emblematica
			\centering
			\fbox{\parbox{0.9\textwidth}{\centering\vspace{2.5cm}
					[Collage fauna emblematica]
					\vspace{2.5cm}}}
		\end{column}
	\end{columns}
	\note{
		Destacar la extraordinaria concentración de especies. Mencionar que el
		yaguareté, el tapir y el águila harpía son especies bandera que requieren
		grandes territorios y su presencia indica un ecosistema saludable.
	}
\end{frame}

\begin{frame}{El Yaguarete: Especie Bandera}
	\begin{columns}
		\begin{column}{0.5\textwidth}
			\begin{itemize}
				\item \textbf{Monumento Natural} provincial (1988) y nacional (2001)
				\item Censo 2024: $\sim$\num{84} individuos en el Corredor Verde \footcite{vidasilvestre2024censo}
				\item Menos de \num{250} adultos en toda Argentina
				\item \textbf{En peligro critico} de extincion
			\end{itemize}

			\vspace{0.5em}
			Su presencia es \textbf{indicador clave} del estado de salud del ecosistema
		\end{column}
		\begin{column}{0.5\textwidth}
			% TODO: Foto de yaguarete (camara trampa)
			\centering
			\fbox{\parbox{0.9\textwidth}{\centering\vspace{2cm}
					[Foto yaguarete\\camara trampa]
					\vspace{2cm}}}
		\end{column}
	\end{columns}
	\note{
		El yaguareté es la especie más emblemática. Con solo \num{84} individuos en el
		Corredor Verde según el censo 2024, está en peligro crítico. Su
		conservación es prioritaria y requiere monitoreo constante para proteger
		tanto a la especie como su hábitat.
	}
\end{frame}

\begin{frame}{Importancia Ecologica}
	\textbf{Servicios ecosistemicos:}
	\begin{itemize}
		\item Regulacion del ciclo hidrologico
		\item Secuestro de carbono
		\item Proteccion contra la erosion del suelo
		\item Regulacion climatica regional
	\end{itemize}

	\vspace{1em}
	\textbf{Marco legal:} Ley 26.331 de Proteccion de Bosques Nativos \footcite{ley26331bosques}
	\begin{description}
		\item[Categoria I (Rojo):] Muy alto valor - no se transforma
		\item[Categoria II (Amarillo):] Mediano valor - uso sostenible
		\item[Categoria III (Verde):] Bajo valor - puede transformarse
	\end{description}
	\note{
		Además del valor biológico, el bosque provee servicios ecosistémicos
		críticos. La Ley 26.331 establece categorías de protección. En Misiones,
		la mayoría del territorio está en categorías I y II, lo que limita la
		transformación del bosque.
	}
\end{frame}

\begin{frame}{Amenazas a la Conservación}
	% TODO: Imagen ilustrativa de las amenazas
	\centering
	\fbox{\parbox{0.6\textwidth}{\centering\vspace{1.5cm}
			[Imagen amenazas a la conservacion]
			\vspace{1.5cm}}}

	\begin{itemize}
		\item Deforestacion y fragmentacion del habitat
		\item Caza furtiva y trafico de fauna
		\item Intrusion en areas protegidas
		\item Recursos limitados para vigilancia
	\end{itemize}
	\note{
		Estas son las principales amenazas que enfrentan las áreas protegidas. La
		combinación de estas presiones hace urgente contar con sistemas de
		monitoreo efectivos.
	}
\end{frame}

\begin{frame}{Deforestacion}
	\begin{columns}
		\begin{column}{0.5\textwidth}
			\textbf{1990-2020:} $\sim$\qty{130000}{ha} perdidas solo en el Corredor Verde \footcite{fauba2024corredorverde}

			\vspace{0.5em}
			\begin{itemize}
				\item \qty{77}{\percent} en parcelas $<$ \qty{50}{ha}
				\item Ocupaciones para cultivos de subsistencia
				\item Tala ilegal de madera noble
			\end{itemize}

			\vspace{0.5em}
			\textbf{2025:} Reduccion del \qty{18}{\percent} respecto al promedio historico \footcite{ecologiamisiones2025deforestacion}
		\end{column}
		\begin{column}{0.5\textwidth}
			% TODO: Grafico/mapa de deforestacion
			\centering
			\fbox{\parbox{0.9\textwidth}{\centering\vspace{2cm}
					[Mapa deforestacion\\o grafico temporal]
					\vspace{2cm}}}
		\end{column}
	\end{columns}
	\note{
		La deforestación ha sido significativa: \num{130000} hectáreas perdidas en \num{30}
		años. Aunque hay mejoras recientes (\qty{18}{\percent} de reducción en 2025), la presión
		continúa principalmente por ocupaciones ilegales en parcelas pequeñas.
	}
\end{frame}

\begin{frame}{Fragmentacion del Habitat}
	\begin{alertblock}{Consecuencias}
		\begin{itemize}
			\item Poblaciones aisladas geneticamente
			\item Desplazamiento por areas no protegidas
			\item Conflictos con actividades humanas
			\item Atropellamientos en rutas
		\end{itemize}
	\end{alertblock}

	% TODO: Diagrama de fragmentacion
	\centering
	\fbox{\parbox{0.7\textwidth}{\centering\vspace{1.5cm}
			[Diagrama fragmentacion del habitat]
			\vspace{1.5cm}}}
	\note{
		La fragmentación aísla poblaciones, reduce la diversidad genética y obliga
		a los animales a cruzar zonas no protegidas, donde enfrentan riesgos
		adicionales como atropellamientos.
	}
\end{frame}

\begin{frame}{Caza Furtiva}
	\begin{columns}
		\begin{column}{0.55\textwidth}
			\textbf{Dos dimensiones:}
			\begin{enumerate}
				\item \textbf{Cultural/subsistencia:} Residentes locales
				\item \textbf{Economica/organizada:} Trafico de fauna
			\end{enumerate}

			\vspace{0.5em}
			\textbf{Zonas criticas:}
			\begin{itemize}
				\item Frontera con Brasil
				\item Reserva de Biosfera Yaboti
				\item Parques provinciales Pinalito, Urugua-i
			\end{itemize}

			\vspace{0.5em}
			\textbf{Especies afectadas:} Tapir, paca, corzuelas, tucanes, loros
		\end{column}
		\begin{column}{0.45\textwidth}
			% TODO: Imagen relacionada a caza furtiva
			\centering
			\fbox{\parbox{0.9\textwidth}{\centering\vspace{2cm}
					[Imagen problematica\\caza furtiva]
					\vspace{2cm}}}
		\end{column}
	\end{columns}
	\note{
		La caza furtiva tiene dos caras: la cultural/subsistencia de pobladores
		locales y el tráfico organizado. Las zonas fronterizas son especialmente
		críticas. Especies como el tapir y los loros son muy afectadas.
	}
\end{frame}

\begin{frame}{Intrusion en Areas Protegidas}
	\textbf{Actividades ilicitas frecuentes:}
	\begin{itemize}
		\item Pesca ilegal en cursos de agua
		\item Desmonte encubierto para expansion de cultivos
		\item Campamentos de caza con infraestructura permanente
		\item Extraccion de recursos naturales
	\end{itemize}

	\vspace{1em}
	\begin{alertblock}{Problema critico}
		Sin un sistema de \textbf{alerta temprana}, las intrusiones se descubren \textbf{a posteriori} durante patrullajes de rutina, cuando el dano ya fue perpetrado.
	\end{alertblock}
	\note{
		Sin alertas en tiempo real, las intrusiones se detectan tardíamente. Este
		es el problema central que nuestro proyecto busca resolver: transformar el
		monitoreo pasivo en vigilancia activa.
	}
\end{frame}

\begin{frame}{Desafios de la Vigilancia}
	\textbf{\num{780000} hectareas} distribuidas en \textbf{\num{106}+ areas protegidas}

	\vspace{0.5em}
	\begin{description}
		\item[Extension:] Terreno accidentado, vegetacion densa, acceso limitado
		\item[Comunicaciones:] Sin cobertura celular en zonas interiores
		\item[Latencia:] Semanas/meses entre captura de evento y descubrimiento
		\item[Volumen:] Miles de imagenes requieren clasificacion manual
	\end{description}

	% TODO: Foto de terreno/vegetacion densa
	\centering
	\fbox{\parbox{0.5\textwidth}{\centering\vspace{1.2cm}
			[Foto vegetacion densa]
			\vspace{1.2cm}}}
	\note{
		\num{780000} hectáreas en más de \num{100} áreas protegidas. Sin cobertura celular,
		con terreno difícil y miles de imágenes que clasificar manualmente. Estos
		desafíos motivan nuestra propuesta tecnológica.
	}
\end{frame}

% ==============================================================================
% Sección 2: Sistemas Actuales y Limitaciones
% ==============================================================================
\section{Sistemas Actuales}

\begin{frame}{Camaras Trampa Tradicionales \footcite{steenweg2017scaling}}
	\begin{columns}
		\begin{column}{0.5\textwidth}
			\textbf{Componentes:}
			\begin{itemize}
				\item Sensor PIR (movimiento)
				\item Camara digital
				\item Almacenamiento SD
				\item Iluminacion IR (nocturna)
				\item Baterias AA
			\end{itemize}

			\textbf{Ventajas:}
			\begin{itemize}
				\item Alta autonomia
				\item Bajo costo inicial
				\item Robustez probada
			\end{itemize}
		\end{column}
		\begin{column}{0.5\textwidth}
			% TODO: Foto camara trampa tradicional
			\centering
			\fbox{\parbox{0.9\textwidth}{\centering\vspace{2cm}
					[Foto camara trampa tradicional]
					\vspace{2cm}}}
		\end{column}
	\end{columns}
	\note{
		Las cámaras trampa tradicionales son la herramienta estándar. Sensor PIR
		detecta movimiento, guarda en SD. Ventajas: alta autonomía y robustez.
		Pero tienen limitaciones importantes.
	}
\end{frame}

\begin{frame}{Limitaciones de Camaras Tradicionales}
	\begin{alertblock}{Brechas criticas}
		\begin{description}
			\item[Latencia:] Imagenes almacenadas localmente por semanas/meses
			\item[Sin alertas:] Informacion fluye solo hacia centros de analisis
			\item[Manual:] \qty{80}{\percent}+ de imagenes son vacias o irrelevantes \footcite{tabak2019machine}
			\item[Mantenimiento:] Visitas periodicas para baterias y tarjetas
		\end{description}
	\end{alertblock}

	\vspace{0.5em}
	\centering
	\textit{Para cuando se detecta una intrusion,\\los responsables ya estan lejos del area.}
	\note{
		Las brechas críticas: latencia de semanas/meses, sin alertas, \qty{80}{\percent} de
		imágenes vacías requieren revisión manual. Para vigilancia antifurtiva,
		esto es inaceptable.
	}
\end{frame}

\begin{frame}{Soluciones Comerciales Celulares}
	\begin{columns}
		\begin{column}{0.55\textwidth}
			\textbf{Ejemplos:} Tactacam REVEAL \footcite{tactacam2024reveal}, Spypoint Flex \footcite{spypoint2024cellular}

			\vspace{0.5em}
			\textbf{Caracteristicas:}
			\begin{itemize}
				\item Envio via redes 3G/4G/LTE
				\item App movil propietaria
				\item Disparo rapido ($<$ \qty{0.5}{\second})
				\item Vision nocturna IR
			\end{itemize}
		\end{column}
		\begin{column}{0.45\textwidth}
			% TODO: Imagen camara celular comercial
			\centering
			\fbox{\parbox{0.9\textwidth}{\centering\vspace{2cm}
					[Camara celular comercial]
					\vspace{2cm}}}
		\end{column}
	\end{columns}
	\note{
		Existen cámaras celulares comerciales como Tactacam y Spypoint que envían
		imágenes por 3G/4G. Ofrecen alertas rápidas pero dependen de cobertura
		celular.
	}
\end{frame}

\begin{frame}{Limitaciones de Soluciones Comerciales}
	\begin{block}{Dependencia de infraestructura}
		\begin{itemize}
			\item \textbf{Sin cobertura celular} en interior de reservas
			\item En selva densa, la senal es inexistente
		\end{itemize}
	\end{block}

	\begin{block}{Costos elevados}
		\begin{itemize}
			\item Alto costo de adquisicion por unidad
			\item Suscripciones mensuales: \qtyrange{5}{15}{\$} USD por camara
			\item Servicios en la nube propietarios
		\end{itemize}
	\end{block}

	\begin{block}{Escalabilidad limitada}
		\begin{itemize}
			\item Costo prohibitivo para despliegues masivos
			\item Especialmente en paises en vias de desarrollo
		\end{itemize}
	\end{block}
	\note{
		Tres problemas: 1) Sin cobertura celular en selva, 2) Costos altos de
		adquisición y suscripciones, 3) No escalan para despliegues masivos en
		países con recursos limitados.
	}
\end{frame}

\begin{frame}{Trabajos Relacionados}
	\begin{description}
		\item[Barrero \& Schmunck (UNaM, 2023) \footcite{barrero2023microcamara}:] Microcamara de vigilancia para fauna - bases para soluciones de bajo costo

		\item[Whytock et al. (2023) \footcite{whytock2023iridium}:] Camaras trampa con IA + alertas satelitales Iridium - costos operativos muy altos

		\item[AiCatcher (Mallya, 2019) \footcite{mallya2019aicatcher}:] Raspberry Pi + LoRa + inferencia en borde - alto consumo energetico
	\end{description}

	\vspace{0.5em}
	\begin{alertblock}{Brecha identificada}
		Oportunidad para un sistema que combine \textbf{bajo costo}, \textbf{independencia de infraestructura celular}, y \textbf{alertas en tiempo real}.
	\end{alertblock}
	\note{
		Trabajo previo en UNaM sentó bases. Whytock usó satélite Iridium con
		costos altísimos. AiCatcher tiene alto consumo. Hay una brecha: bajo costo
		+ independencia de celular + alertas en tiempo real.
	}
\end{frame}

\begin{frame}{Analisis Comparativo}
	\begin{table}
		\scriptsize
		\centering
		\begin{tabular}{lcccc}
			\hline
			\textbf{Caracteristica} & \textbf{Tradicional} & \textbf{Celular} & \textbf{Satelital} & \textbf{LoRa} \\
			\hline
			Alertas tiempo real     & No                   & Si               & Si                 & Si            \\
			Infraestructura         & Ninguna              & Operador         & Satelite           & Gateway       \\
			Procesamiento IA        & Post-hoc             & No/Limitado      & Edge               & Edge          \\
			Costo adquisicion       & Bajo                 & Medio-Alto       & Muy alto           & Medio         \\
			Costo operativo         & Bajo                 & Alto             & Muy alto           & Bajo          \\
			Autonomia               & Muy alta             & Media            & Baja               & Baja          \\
			\hline
		\end{tabular}
	\end{table}

	\vspace{0.5em}
	\centering
	\textit{Las alertas inmediatas estan condicionadas por\\elevados costos o dependencia de terceros.}
	\note{
		Esta tabla resume las opciones disponibles. Ninguna combina bajo costo,
		independencia de infraestructura y alertas en tiempo real. Esa es la
		oportunidad que identificamos.
	}
\end{frame}

% ==============================================================================
% Sección 3: Motivación y Objetivos
% ==============================================================================
\section{Motivacion y Objetivos}

\begin{frame}{Motivacion del Proyecto}
	\textbf{Transformar el paradigma:}

	\begin{center}
		Monitoreo \textbf{pasivo} $\longrightarrow$ Vigilancia \textbf{activa e inteligente}
	\end{center}

	\vspace{1em}
	\textbf{Un sistema que:}
	\begin{itemize}
		\item No solo capture imagenes, sino que las \textbf{transmita en tiempo real}
		\item Las \textbf{analice automaticamente} mediante IA
		\item Genere \textbf{alertas inmediatas} ante eventos de interes
		\item Sea \textbf{accesible y de bajo costo}
		\item Use \textbf{hardware economico} y \textbf{software de codigo abierto}
	\end{itemize}
	\note{
		Nuestra motivación: pasar del monitoreo pasivo (revisar fotos semanas
		después) a vigilancia activa con alertas inmediatas. Un sistema accesible,
		de bajo costo, con código abierto.
	}
\end{frame}

\begin{frame}{Oportunidad Tecnologica}
	\textbf{Convergencia de tres desarrollos:}

	\vspace{0.5em}
	\begin{block}{1. Microcontroladores de bajo costo}
		ESP32: WiFi, Bluetooth, bajo consumo. Costo: $<$ \qty{10}{\$} USD
	\end{block}

	\begin{block}{2. Redes mesh autoorganizadas}
		ESP-MESH: Extension de cobertura sin infraestructura celular
	\end{block}

	\begin{block}{3. IA para vision por computadora}
		SpeciesNet, MegaDetector: Modelos de codigo abierto para clasificacion automatica de fauna
	\end{block}
	\note{
		Tres tecnologías convergen: ESP32 a menos de \num{10} dólares con WiFi integrado,
		redes mesh que extienden cobertura sin infraestructura, y modelos de IA de
		código abierto como SpeciesNet.
	}
\end{frame}

\begin{frame}{Objetivo General}
	\begin{alertblock}{Objetivo}
		Demostrar la \textbf{viabilidad tecnica} de un sistema de monitoreo de \textbf{bajo costo} para areas protegidas, basado en:
		\begin{itemize}
			\item Nodos de camara con conectividad \textbf{mesh}
			\item Clasificacion automatica de imagenes mediante \textbf{IA}
		\end{itemize}
	\end{alertblock}

	\vspace{1em}
	\textbf{Reduccion del tiempo de procesamiento:}
	\begin{center}
		\textbf{Dias/semanas} $\longrightarrow$ \textbf{Minutos}
	\end{center}
	\note{
		El objetivo central es demostrar viabilidad técnica. No comercialización,
		sino prueba de concepto funcional. El cambio clave: reducir el tiempo de
		procesamiento de días a minutos.
	}
\end{frame}

\begin{frame}{Objetivos Especificos}
	\begin{enumerate}
		\item Disenar e implementar un \textbf{nodo de camara autonomo} basado en ESP32

		\item Desarrollar una \textbf{red mesh autoorganizada} con ESP-MESH

		\item Implementar un \textbf{servidor de procesamiento} con IA

		\item Integrar \textbf{SpeciesNet/MegaDetector} para clasificacion

		\item \textbf{Validar el funcionamiento} en condiciones reales
	\end{enumerate}
	\note{
		Cinco objetivos específicos que abarcan hardware, red, servidor, IA y
		validación. Cada uno corresponde a un componente del sistema que
		desarrollamos.
	}
\end{frame}

\begin{frame}{Alcance del Proyecto}
	\begin{columns}
		\begin{column}{0.5\textwidth}
			\textbf{Dentro del alcance:}
			\begin{itemize}
				\item Nodos ESP32-CAM + sensor PIR
				\item Red mesh funcional
				\item Servidor de recepcion y clasificacion
				\item Bot de Telegram para alertas
				\item Pruebas en laboratorio y campo
			\end{itemize}
		\end{column}
		\begin{column}{0.5\textwidth}
			\textbf{Fuera del alcance:}
			\begin{itemize}
				\item Despliegue en selva densa
				\item Carcasas con grado IP
				\item Entrenamiento de modelos especificos
				\item Certificacion comercial
			\end{itemize}
		\end{column}
	\end{columns}
	\note{
		Definimos claramente qué sí y qué no incluye el proyecto. Prototipo
		funcional para pruebas, no producto comercial. Sin despliegue en selva
		real ni carcasas industriales.
	}
\end{frame}

\begin{frame}{Limitaciones Conocidas}
	\begin{description}
		\item[Operacion diurna:] Lente con filtro IR, sin vision nocturna
		\item[Sensor PIR:] Optimizado para humanos, puede no detectar fauna pequena
		\item[Resolucion:] Limitada por tamano maximo de paquete mesh ($<$ \qty{8}{\kilo\byte})
		\item[Consumo:] WiFi consume mas que camaras trampa comerciales optimizadas
	\end{description}

	\vspace{0.5em}
	\begin{block}{Nota sobre ``tiempo real''}
		No es tiempo real deterministico (milisegundos), sino \textbf{operativo}: de dias/semanas a \textbf{minutos}.
	\end{block}
	\note{
		Transparencia sobre limitaciones: sin visión nocturna, PIR optimizado para
		humanos, resolución limitada por tamaño de paquete mesh. Tiempo real
		significa minutos, no milisegundos.
	}
\end{frame}

% ==============================================================================
% Sección 4: Marco Teórico
% ==============================================================================
\section{Marco Teorico}

\begin{frame}{Funcionamiento de Camaras Trampa}
	\textbf{Ciclo de operacion:}
	\begin{enumerate}
		\item \textbf{Reposo:} Solo sensor PIR activo (bajo consumo)
		\item \textbf{Deteccion:} Cambio en radiacion IR
		\item \textbf{Activacion:} Despertar camara (trigger time: \qtyrange{0.1}{2}{\second})
		\item \textbf{Captura:} Foto/video + metadatos
		\item \textbf{Almacenamiento:} Compresion JPEG, grabacion en SD
		\item \textbf{Retorno al reposo}
	\end{enumerate}

	% TODO: Diagrama del ciclo
	\centering
	\fbox{\parbox{0.6\textwidth}{\centering\vspace{1cm}
			[Diagrama ciclo operacion]
			\vspace{1cm}}}
	\note{
		Ciclo básico de una cámara trampa: reposo con PIR activo, detección de
		movimiento, captura, almacenamiento, retorno al reposo. Nuestro sistema
		agrega transmisión y procesamiento.
	}
\end{frame}

\begin{frame}{Internet de las Cosas (IoT)}
	\textbf{Arquitectura de tres capas:}

	% TODO: Diagrama arquitectura IoT
	\centering
	\fbox{\parbox{0.7\textwidth}{\centering\vspace{2cm}
			[Diagrama arquitectura IoT - 3 capas]
			\vspace{2cm}}}

	\begin{enumerate}
		\item \textbf{Percepcion:} Sensores, camaras $\rightarrow$ ESP32-CAM + PIR
		\item \textbf{Red:} Transmision de datos $\rightarrow$ ESP-MESH
		\item \textbf{Aplicacion:} Procesamiento, alertas $\rightarrow$ Django + SpeciesNet
	\end{enumerate}
	\note{
		Arquitectura IoT clásica de tres capas. Percepción: nuestros ESP32-CAM.
		Red: ESP-MESH. Aplicación: servidor Django con SpeciesNet. Esta estructura
		guía todo el diseño.
	}
\end{frame}

\begin{frame}{Protocolos de Comunicacion}
	\begin{table}
		\small
		\centering
		\begin{tabular}{lccc}
			\hline
			\textbf{Protocolo} & \textbf{Alcance} & \textbf{Ancho banda} & \textbf{Consumo} \\
			\hline
			BLE                & $\sim$\qty{100}{\metre}       & Bajo                 & Muy bajo         \\
			ZigBee             & $\sim$\qty{100}{\metre}       & Bajo                 & Bajo             \\
			LoRa               & km               & Muy bajo             & Bajo             \\
			WiFi               & $\sim$\qty{100}{\metre}       & \textbf{Alto}        & Alto             \\
			\hline
		\end{tabular}
	\end{table}

	\vspace{0.5em}
	\textbf{LoRa:} Excelente alcance, pero ancho de banda \textbf{insuficiente} para imagenes

	\vspace{0.5em}
	\textbf{WiFi Mesh:} Permite transmision de imagenes + extension de cobertura
	\note{
		Comparamos protocolos. LoRa tiene gran alcance pero poco ancho de banda para
		imágenes. WiFi tiene alto ancho de banda pero alcance limitado. Solución:
		WiFi Mesh extiende cobertura.
	}
\end{frame}

\begin{frame}{Redes Mesh - ESP-MESH \footcite{espmesh2023}}
	\begin{columns}
		\begin{column}{0.5\textwidth}
			\textbf{Caracteristicas:}
			\begin{itemize}
				\item \textbf{Autoorganizacion:} Seleccion automatica de padre optimo
				\item \textbf{Autocuracion:} Reconexion ante fallos
				\item Topologia de arbol
				\item Hasta \num{6}+ niveles de profundidad
				\item Cientos de nodos
			\end{itemize}
		\end{column}
		\begin{column}{0.5\textwidth}
			% TODO: Diagrama topologia mesh
			\centering
			\fbox{\parbox{0.9\textwidth}{\centering\vspace{2cm}
					[Diagrama topologia mesh]
					\vspace{2cm}}}
		\end{column}
	\end{columns}
	\note{
		ESP-MESH de Espressif: red autoorganizada donde los nodos eligen el mejor
		padre, se autocura ante fallos. Topología de árbol con hasta 6 niveles.
		Soporta cientos de nodos.
	}
\end{frame}

\begin{frame}{Deteccion de Objetos con YOLO \footcite{redmon2016yolo}}
	\begin{columns}
		\begin{column}{0.5\textwidth}
			\textbf{You Only Look Once}
			\begin{itemize}
				\item Deteccion en una sola pasada
				\item Bounding boxes + probabilidades
				\item Balance velocidad/precision
				\item YOLOv5: PyTorch, multiples variantes \footcite{jocher2020yolov5}
			\end{itemize}
		\end{column}
		\begin{column}{0.5\textwidth}
			% TODO: Ejemplo deteccion YOLO
			\centering
			\fbox{\parbox{0.9\textwidth}{\centering\vspace{2cm}
					[Ejemplo deteccion YOLO\\con bounding boxes]
					\vspace{2cm}}}
		\end{column}
	\end{columns}
	\note{
		YOLO revolucionó la detección de objetos con una sola pasada por la imagen.
		Rápido y preciso. YOLOv5 en PyTorch es la base de MegaDetector que usamos.
	}
\end{frame}

\begin{frame}{MegaDetector \footcite{beery2019megadetector}}
	\textbf{Microsoft AI for Earth}

	\vspace{0.5em}
	Modelo especializado para camaras trampa que detecta:
	\begin{itemize}
		\item \textbf{Animales} (sin distincion de especie)
		\item \textbf{Personas}
		\item \textbf{Vehiculos}
	\end{itemize}

	\vspace{0.5em}
	\textbf{Ventajas:}
	\begin{itemize}
		\item Altamente robusto y generalizable
		\item Adoptado por \num{60}+ organizaciones de conservacion \footcite{velez2022platforms}
		\item Reduce hasta \qty{90}{\percent} el tiempo de procesamiento
	\end{itemize}
	\note{
		MegaDetector de Microsoft: detecta animales, personas y vehículos sin
		clasificar especies. Altamente robusto, usado por más de \num{60}
		organizaciones. Reduce \qty{90}{\percent} del tiempo de revisión.
	}
\end{frame}

\begin{frame}{SpeciesNet - Google \footcite{gadot2024crop}}
	\textbf{Pipeline de dos fases:}

	\begin{enumerate}
		\item \textbf{Deteccion:} Identifica regiones de interes (animales, personas, vehiculos)

		\item \textbf{Clasificacion taxonomica:} Para animales detectados
		      \begin{itemize}
			      \item Familia
			      \item Genero
			      \item Especie
		      \end{itemize}
	\end{enumerate}

	\vspace{0.5em}
	\textbf{Entrenado con millones de imagenes} de camaras trampa a nivel global

	% TODO: Ejemplo de clasificacion
	\centering
	\fbox{\parbox{0.5\textwidth}{\centering\vspace{1cm}
			[Ejemplo clasificacion taxonomica]
			\vspace{1cm}}}
	\note{
		SpeciesNet de Google va más allá: detecta Y clasifica taxonómicamente.
		Entrenado con millones de imágenes de cámaras trampa. Devuelve familia,
		género y especie con niveles de confianza.
	}
\end{frame}

% ==============================================================================
% Sección 5: Arquitectura del Sistema
% ==============================================================================
\section{Arquitectura del Sistema}

\begin{frame}{Vision General}
	% TODO: Diagrama arquitectura general
	\centering
	\fbox{\parbox{0.9\textwidth}{\centering\vspace{3cm}
			[Diagrama arquitectura general del sistema]\\
			Nodos camara $\rightarrow$ Red Mesh $\rightarrow$ Nodo Raiz $\rightarrow$ Servidor $\rightarrow$ Telegram
			\vspace{3cm}}}
	\note{
		Visión general del sistema: nodos de cámara capturan imágenes, las envían por
		red mesh al nodo raíz, que las reenvía al servidor para procesamiento con IA
		y notificación por Telegram.
	}
\end{frame}

\begin{frame}{Flujo de Datos}
	\begin{enumerate}
		\item Sensor PIR detecta movimiento $\rightarrow$ interrupcion
		\item Nodo captura imagen JPEG ($<$ \qty{8}{\kilo\byte})
		\item Transmision via red mesh hasta nodo raiz
		\item Nodo raiz envia HTTP POST al servidor
		\item Servidor procesa con SpeciesNet
		\item Si hay deteccion: alerta via Telegram
		\item Almacenamiento en base de datos
	\end{enumerate}
	\note{
		Flujo paso a paso: PIR detecta, nodo captura JPEG menor a \qty{8}{\kilo\byte}, transmite por
		mesh, nodo raíz envía HTTP POST, servidor procesa con SpeciesNet, si hay
		detección envía alerta Telegram.
	}
\end{frame}

\begin{frame}{Nodo de Camara - Hardware}
	\begin{columns}
		\begin{column}{0.5\textwidth}
			\textbf{Componentes:}
			\begin{itemize}
				\item ESP32-CAM (AI-Thinker)
				\item Camara OV2640 (\num{2}~MP)
				\item Sensor PIR HC-SR501
				\item Buck converter LM2596
				\item Baterias 18650 (2S, \qty{7.4}{\volt})
			\end{itemize}

			\vspace{0.5em}
			\textbf{Resolucion:} \num{640} x \num{480} (VGA)
		\end{column}
		\begin{column}{0.5\textwidth}
			% TODO: Foto del nodo ensamblado
			\centering
			\fbox{\parbox{0.9\textwidth}{\centering\vspace{2cm}
					[Foto nodo camara ensamblado]
					\vspace{2cm}}}
		\end{column}
	\end{columns}
	\note{
		Hardware del nodo: ESP32-CAM de AI-Thinker con cámara OV2640, sensor PIR
		HC-SR501, buck converter para alimentación desde baterías 18650. Resolución
		VGA suficiente para clasificación.
	}
\end{frame}

\begin{frame}{Nodo de Camara - Firmware}
	\textbf{Desarrollado con ESP-IDF + ESP-MDF}

	\vspace{0.5em}
	\textbf{Funcionalidades:}
	\begin{itemize}
		\item Inicializacion de camara OV2640
		\item Interrupcion GPIO para sensor PIR
		\item Conexion a red ESP-MESH como nodo hijo
		\item Captura y compresion JPEG
		\item Almacenamiento en microSD (respaldo)
		\item Transmision via Mwifi
		\item Modo power-save para ahorro de energia
	\end{itemize}
	\note{
		Firmware desarrollado con ESP-IDF y ESP-MDF. Funcionalidades: inicialización
		de cámara, interrupción PIR, conexión mesh, captura JPEG, respaldo en SD,
		transmisión Mwifi, modo power-save. Límite de paquete mesh: \qty{8}{\kilo\byte}.
	}
\end{frame}

\begin{frame}{Carcasa Impresa en 3D}
	\begin{columns}
		\begin{column}{0.5\textwidth}
			\textbf{Diseno basado en modelo CC \num{4.0}}

			\vspace{0.5em}
			\textbf{Caracteristicas:}
			\begin{itemize}
				\item Material: PLA
				\item Apertura para lente
				\item Domo para sensor PIR
				\item Espacio para buck converter
				\item Acceso a tarjeta SD
				\item Puntos de montaje
			\end{itemize}

			\vspace{0.5em}
			\textit{Disponible en Printables}
		\end{column}
		\begin{column}{0.5\textwidth}
			% TODO: Render/foto carcasa
			\centering
			\fbox{\parbox{0.9\textwidth}{\centering\vspace{2cm}
					[Render/foto carcasa 3D]
					\vspace{2cm}}}
		\end{column}
	\end{columns}
	\note{
		Carcasa impresa en 3D basada en diseño CC 4.0. PLA, aperturas para lente y
		PIR, espacio para electrónica, acceso a SD, puntos de montaje. Disponible
		en Printables para que cualquiera pueda fabricarla.
	}
\end{frame}

\begin{frame}{Nodo Raiz}
	\begin{columns}
		\begin{column}{0.55\textwidth}
			\textbf{Hardware:} ESP32 DevKit V1 (sin camara)

			\vspace{0.5em}
			\textbf{Rol:} Puerta de enlace Mesh $\leftrightarrow$ Internet

			\vspace{0.5em}
			\textbf{Funciones:}
			\begin{itemize}
				\item Raiz de la red ESP-MESH
				\item Recepcion de imagenes de nodos hijos
				\item Conexion WiFi a router
				\item Cliente HTTP para envio al servidor
				\item Sincronizacion de tiempo
			\end{itemize}
		\end{column}
		\begin{column}{0.45\textwidth}
			% TODO: Diagrama flujo nodo raiz
			\centering
			\fbox{\parbox{0.9\textwidth}{\centering\vspace{2cm}
					[Diagrama nodo raiz]
					\vspace{2cm}}}
		\end{column}
	\end{columns}
	\note{
		Nodo raíz: ESP32 DevKit sin cámara, actúa como gateway entre mesh e
		internet. Recibe imágenes de nodos hijos, las envía por HTTP POST al
		servidor. Siempre conectado a router WiFi.
	}
\end{frame}

\begin{frame}{Servidor de Aplicacion}
	\textbf{Arquitectura containerizada con Docker Compose:}

	\vspace{0.5em}
	\begin{description}
		\item[Django:] Backend web, recepcion de imagenes
		\item[SpeciesNet:] Servicio de inferencia (LitServe)
		\item[PostgreSQL:] Base de datos
		\item[Telegram Bot:] Notificaciones
	\end{description}

	% TODO: Diagrama Docker Compose
	\centering
	\fbox{\parbox{0.6\textwidth}{\centering\vspace{1.5cm}
			[Diagrama Docker Compose]
			\vspace{1.5cm}}}
	\note{
		Servidor containerizado con Docker Compose: Django para backend,
		SpeciesNet para inferencia con LitServe, PostgreSQL para datos, bot de
		Telegram para notificaciones. Fácil de desplegar.
	}
\end{frame}

\begin{frame}{Sistema de Alertas}
	\begin{columns}
		\begin{column}{0.5\textwidth}
			\textbf{Bot de Telegram}

			\vspace{0.5em}
			Cada notificacion incluye:
			\begin{itemize}
				\item Imagen original
				\item Imagen anotada (bounding boxes)
				\item Cantidad y confianza de detecciones
				\item Clasificaciones taxonomicas
			\end{itemize}

			\vspace{0.5em}
			\textbf{Comandos:}
			\begin{itemize}
				\item \texttt{/start} - Registro
				\item \texttt{/last} - Ultima imagen
			\end{itemize}
		\end{column}
		\begin{column}{0.5\textwidth}
			% TODO: Screenshot Telegram
			\centering
			\fbox{\parbox{0.9\textwidth}{\centering\vspace{2.5cm}
					[Screenshot notificacion Telegram]
					\vspace{2.5cm}}}
		\end{column}
	\end{columns}
	\note{
		Bot de Telegram envía notificaciones con imagen original, imagen anotada
		con bounding boxes, detecciones con confianza y clasificaciones
		taxonómicas. Comandos /start y /last para interacción.
	}
\end{frame}

% ==============================================================================
% Sección 6: Metodología
% ==============================================================================
\section{Metodologia}

\begin{frame}{Enfoque de Desarrollo}
	\textbf{Iterativo e incremental}

	\vspace{0.5em}
	\begin{itemize}
		\item Cada componente desarrollado, probado y refinado individualmente
		\item Identificacion temprana de problemas
		\item Metodologia agil \textbf{Kanban}
	\end{itemize}

	\vspace{0.5em}
	\textbf{Control de versiones:} GitHub (4 repositorios)

	\vspace{0.5em}
	\textbf{Despliegue continuo:} GitHub Actions
	\begin{itemize}
		\item Build automatico de imagenes Docker
		\item Publicacion en GitHub Container Registry
	\end{itemize}
	\note{
		Enfoque iterativo: cada componente se desarrolló y probó individualmente.
		Metodología Kanban para gestión ágil. 4 repositorios en GitHub. CI/CD con
		GitHub Actions para imágenes Docker.
	}
\end{frame}

\begin{frame}{Etapas del Desarrollo}
	% TODO: Diagrama Gantt simplificado
	\centering
	\fbox{\parbox{0.8\textwidth}{\centering\vspace{2cm}
			[Diagrama Gantt - etapas del desarrollo]
			\vspace{2cm}}}

	\begin{enumerate}
		\item Investigacion y diseno
		\item Desarrollo del firmware
		\item Desarrollo del servidor
		\item Diseno y fabricacion de hardware
		\item Integracion
		\item Pruebas y validacion
	\end{enumerate}
	\note{
		6 etapas principales: investigación, firmware, servidor, hardware 3D,
		integración y pruebas. El desarrollo fue paralelo cuando fue posible para
		optimizar tiempos.
	}
\end{frame}

\begin{frame}{Herramientas Utilizadas}
	\begin{columns}
		\begin{column}{0.5\textwidth}
			\textbf{Desarrollo:}
			\begin{itemize}
				\item VS Code
				\item ESP-IDF + ESP-MDF
				\item Django
				\item python-telegram-bot
				\item Docker / Docker Compose
			\end{itemize}
		\end{column}
		\begin{column}{0.5\textwidth}
			\textbf{Diseno 3D:}
			\begin{itemize}
				\item Autodesk Fusion 360
				\item PrusaSlicer
				\item Impresora Creality Ender 3
			\end{itemize}
		\end{column}
	\end{columns}
	\note{
		Herramientas: VS Code, ESP-IDF/ESP-MDF para firmware, Django para servidor,
		Docker para despliegue. Diseño 3D con Fusion 360, slicing con
		PrusaSlicer, impresión en Ender 3.
	}
\end{frame}

\begin{frame}{Desafios Enfrentados}
	\begin{alertblock}{Desafio 1: Limite de tamano de paquete}
		ESP-MDF limita paquetes a \qty{8}{\kilo\byte} $\rightarrow$ ajustar compresion JPEG
	\end{alertblock}

	\begin{alertblock}{Desafio 2: Consumo energetico}
		WiFi consume significativamente $\rightarrow$ modo power-save de Mwifi
	\end{alertblock}

	\begin{alertblock}{Desafio 3: Estabilidad de red mesh}
		Reconexiones frecuentes $\rightarrow$ tuning de parametros de conexion
	\end{alertblock}

	\begin{alertblock}{Desafio 4: Tiempo de inferencia}
		SpeciesNet lento en CPU $\rightarrow$ soporte para GPU (nvidia-container-toolkit)
	\end{alertblock}
	\note{
		Desafíos principales: 1) Límite \qty{8}{\kilo\byte} por paquete mesh, 2) Consumo energético
		del WiFi, 3) Estabilidad de reconexiones, 4) Lentitud de inferencia en CPU.
		Cada uno tuvo su solución.
	}
\end{frame}

% ==============================================================================
% Sección 7: Pruebas y Resultados
% ==============================================================================
\section{Pruebas y Resultados}

\begin{frame}{Configuracion de Pruebas}
	\begin{columns}
		\begin{column}{0.5\textwidth}
			\textbf{Hardware:}
			\begin{itemize}
				\item 3 nodos de camara
				\item 1 nodo raiz
				\item Baterias 18650 (2S)
				\item Router WiFi domestico
			\end{itemize}
		\end{column}
		\begin{column}{0.5\textwidth}
			\textbf{Software:}
			\begin{itemize}
				\item ESP-IDF v5.x
				\item Docker containers
				\item SpeciesNet + LitServe
				\item PostgreSQL
				\item Tunel VPN (Pangolin)
			\end{itemize}
		\end{column}
	\end{columns}

	% TODO: Foto setup
	\centering
	\fbox{\parbox{0.5\textwidth}{\centering\vspace{1cm}
			[Foto setup de pruebas]
			\vspace{1cm}}}
	\note{
		Configuración de pruebas: 3 nodos cámara, 1 nodo raíz, baterías 18650,
		router doméstico. Software: ESP-IDF v5, Docker, SpeciesNet, PostgreSQL,
		túnel VPN Pangolin para acceso remoto.
	}
\end{frame}

\begin{frame}{Pruebas de Laboratorio}
	\textbf{Validaciones realizadas:}

	\begin{itemize}
		\item Captura de imagen: inicializacion, calidad, tamano
		\item Sensor PIR: distancia, tiempo de respuesta, falsas activaciones
		\item Transmision mesh: conexion, latencia, perdida de paquetes
		\item Clasificacion: deteccion de animales/personas/vehiculos
		\item Alertas: recepcion en Telegram, contenido correcto
	\end{itemize}
	\note{
		Pruebas de laboratorio cubrieron cada componente: captura de imagen, sensor
		PIR, transmisión mesh, clasificación con IA, y recepción de alertas en
		Telegram.
	}
\end{frame}

\begin{frame}{Resultados - Captura de Imagen}
	% TODO: Ejemplo imagen capturada
	\centering
	\fbox{\parbox{0.6\textwidth}{\centering\vspace{2cm}
			[Ejemplo imagen capturada 640x480]
			\vspace{2cm}}}

	\begin{itemize}
		\item Resolucion: 640x480 (VGA)
		\item Formato: JPEG comprimido
		\item Tamano tipico: $<$ \qty{8}{\kilo\byte}
	\end{itemize}
	\note{
		Resultados de captura: resolución VGA suficiente para clasificación.
		Compresión JPEG ajustada para mantener tamaño bajo \qty{8}{\kilo\byte} y cumplir límite de
		paquete mesh.
	}
\end{frame}

\begin{frame}{Resultados - Red Mesh}
	\begin{table}
		\centering
		\begin{tabular}{lc}
			\hline
			\textbf{Metrica}               & \textbf{Valor} \\
			\hline
			Latencia promedio              & [X] \unit{\second}          \\
			Distancia max (linea de vista) & [X] \unit{\metre}          \\
			Distancia max (con obstaculos) & [X] \unit{\metre}          \\
			Tasa perdida de paquetes       & $<$ \qty{5}{\percent}         \\
			Tiempo de reconexion           & [X] \unit{\second}          \\
			\hline
		\end{tabular}
	\end{table}
	\note{
		Métricas de red mesh: latencia, distancias máximas con y sin obstáculos,
		pérdida de paquetes menor al \qty{5}{\percent}, tiempo de reconexión. Resultados
		consistentes con documentación de Espressif.
	}
\end{frame}

\begin{frame}{Resultados - Deteccion}
	\begin{columns}
		\begin{column}{0.5\textwidth}
			\begin{table}
				\centering
				\begin{tabular}{lc}
					\hline
					\textbf{Categoria} & \textbf{Precision} \\
					\hline
					Animales           & [X]\%              \\
					Personas           & [X]\%              \\
					Vehiculos          & [X]\%              \\
					\hline
				\end{tabular}
			\end{table}

			\vspace{0.5em}
			\textbf{Tiempo de inferencia:}
			\begin{itemize}
				\item CPU: \numrange{1}{5} \unit{\second}/imagen
				\item GPU: \numrange{0.5}{1} \unit{\second}/imagen
			\end{itemize}
		\end{column}
		\begin{column}{0.5\textwidth}
			% TODO: Ejemplo deteccion exitosa
			\centering
			\fbox{\parbox{0.9\textwidth}{\centering\vspace{2cm}
					[Ejemplo deteccion exitosa]
					\vspace{2cm}}}
		\end{column}
	\end{columns}
	\note{
		Resultados de detección con SpeciesNet: precisión por categoría. Tiempo
		de inferencia: \qtyrange{1}{5}{\second} en CPU, \qtyrange{0.5}{1}{\second} con GPU. GPU recomendada
		para producción.
	}
\end{frame}

\begin{frame}{Resultados - Consumo Energetico}
	\begin{table}
		\centering
		\begin{tabular}{lc}
			\hline
			\textbf{Estado}    & \textbf{Consumo (\unit{\milli\ampere})} \\
			\hline
			Modo power-save    & [X]                   \\
			Captura de imagen  & [X]                   \\
			Transmision WiFi   & [X]                   \\
			Promedio ponderado & [X]                   \\
			\hline
		\end{tabular}
	\end{table}

	\vspace{0.5em}
	\textbf{Autonomia estimada:} [X] horas con baterias 18650 (2S, [X]mAh)
	\note{
		Consumo energético: modo power-save consume menos, pero WiFi activo consume
		significativamente. Autonomía estimada depende de frecuencia de
		activaciones.
	}
\end{frame}

\begin{frame}{Pruebas de Campo}
	\begin{columns}
		\begin{column}{0.5\textwidth}
			\textbf{Condiciones:}
			\begin{itemize}
				\item Duracion: [X] horas
				\item 3 nodos desplegados
				\item Distancia entre nodos: [X]m
				\item Condiciones climaticas favorables
			\end{itemize}

			\vspace{0.5em}
			\textbf{Resultados:}
			\begin{itemize}
				\item Imagenes capturadas: [X]
				\item Detecciones exitosas: [X]
				\item Falsas activaciones: [X]
			\end{itemize}
		\end{column}
		\begin{column}{0.5\textwidth}
			% TODO: Foto despliegue campo
			\centering
			\fbox{\parbox{0.9\textwidth}{\centering\vspace{2cm}
					[Foto nodo desplegado en campo]
					\vspace{2cm}}}
		\end{column}
	\end{columns}
	\note{
		Pruebas de campo: despliegue real con 3 nodos durante varias horas.
		Resultados: imágenes capturadas, detecciones exitosas, algunas falsas
		activaciones por movimiento de vegetación.
	}
\end{frame}

% ==============================================================================
% Sección 8: Conclusiones
% ==============================================================================
\section{Conclusiones}

\begin{frame}{Logros Alcanzados}
	\begin{itemize}
		\item[$\checkmark$] Sistema completo de monitoreo con \textbf{hardware de bajo costo}
		\item[$\checkmark$] Reduccion de tiempo de procesamiento: \textbf{dias $\rightarrow$ minutos}
		\item[$\checkmark$] Red mesh funcional con \textbf{ESP-MESH}
		\item[$\checkmark$] Integracion exitosa de \textbf{SpeciesNet/MegaDetector}
		\item[$\checkmark$] Sistema de alertas en tiempo real via \textbf{Telegram}
		\item[$\checkmark$] Carcasa imprimible en 3D
		\item[$\checkmark$] \textbf{Codigo abierto} disponible en GitHub
	\end{itemize}
	\note{
		Resumen de logros: sistema completo funcionando, reducción de tiempo de
		días a minutos, red mesh estable, IA integrada, alertas por Telegram,
		carcasa 3D, todo código abierto.
	}
\end{frame}

\begin{frame}{Aportes del Trabajo}
	\begin{block}{Arquitectura integrada}
		Captura distribuida + transmision mesh + procesamiento IA centralizado
	\end{block}

	\begin{block}{Prototipo de bajo costo}
		Componentes comerciales economicos + disenos 3D listos para fabricacion
	\end{block}

	\begin{block}{Integracion de SpeciesNet}
		Pipeline de clasificacion taxonomica en tiempo operativo
	\end{block}

	\begin{block}{Codigo abierto}
		\num{4} repositorios publicos en GitHub
	\end{block}
	\note{
		Cuatro aportes principales: arquitectura integrada end-to-end, prototipo
		replicable de bajo costo, primera integración de SpeciesNet en tiempo
		operativo, y todo disponible como código abierto.
	}
\end{frame}

\begin{frame}{Comparacion con Objetivos}
	\begin{table}
		\centering
		\begin{tabular}{lcc}
			\hline
			\textbf{Metrica}       & \textbf{Objetivo} & \textbf{Resultado} \\
			\hline
			Latencia de respuesta  & $<$X min          & [X] min            \\
			Precision de deteccion & $>$X\%            & [X]\%              \\
			Autonomia              & $>$X horas        & [X] horas          \\
			Cobertura              & $>$X m            & [X] m              \\
			\hline
		\end{tabular}
	\end{table}
	\note{
		Comparación objetivo vs. resultado para cada métrica clave. Todos los
		objetivos fueron alcanzados o superados dentro de las limitaciones
		conocidas del prototipo.
	}
\end{frame}

\begin{frame}{Trabajos Futuros}
	\begin{description}
		\item[Vision nocturna:] Camaras con capacidad IR
		\item[Conectividad largo alcance:] LoRa, \num{4}G/LTE, satelital
		\item[Optimizacion energetica:] Paneles solares, nodo repetidor dedicado
		\item[Procesamiento en borde:] TinyML en nodos de camara
		\item[Modelos especificos:] Entrenamiento para fauna regional
		\item[Interfaz web:] Mapas, gestion de despliegues, reportes
	\end{description}
	\note{
		Líneas futuras: visión nocturna, conectividad de largo alcance (LoRa, 4G,
		satélite), optimización energética con solar, TinyML en borde, modelos
		específicos para fauna regional, interfaz web.
	}
\end{frame}

\begin{frame}{Recomendaciones}
	\begin{enumerate}
		\item \textbf{Ubicaciones:} Considerar distancia entre nodos y obstaculos
		\item \textbf{Proteccion:} Usar PETG/ASA para despliegues permanentes
		\item \textbf{Baterias:} Dimensionar segun frecuencia de activaciones
		\item \textbf{Sensor PIR:} Evaluar alternativas para fauna pequena
		\item \textbf{Monitoreo:} Implementar seguimiento de estado de nodos
	\end{enumerate}
	\note{
		Recomendaciones prácticas para implementadores: ubicaciones con línea de
		vista, materiales resistentes para exterior, dimensionamiento de baterías,
		alternativas de sensor PIR, monitoreo de estado.
	}
\end{frame}

\begin{frame}{Repositorios}
	\small
	\begin{description}
		\item[mesh-node:] \url{github.com/fabcontigiani/mesh-node-capstone-project}
		\item[root-node:] \url{github.com/fabcontigiani/root-node-capstone-project}
		\item[server:] \url{github.com/fabcontigiani/server-capstone-project}
		\item[wildlife-detection:] \url{github.com/fabcontigiani/wildlife-detection-capstone-project}
	\end{description}

	\vspace{1em}
	\textbf{Modelo 3D carcasa:} Disponible en Printables
	\note{
		Todo el código disponible en GitHub: firmware del nodo mesh, firmware del
		nodo raíz, servidor Django, y contenedor de detección. Carcasa 3D en
		Printables.
	}
\end{frame}

% ==============================================================================
% Cierre
% ==============================================================================

\begin{frame}{Agradecimientos}
	\begin{itemize}
		\item A nuestras familias por el apoyo incondicional
		\item Al Dr. Ing. Sergio Moya por su guía y dedicación como tutor
		\item A la Facultad de Ingeniería de la UNaM
		\item A los docentes que contribuyeron a nuestra formación
		\item A nuestros compañeros de carrera
	\end{itemize}

	\vspace{1em}
	\centering
	\textit{¡Gracias a todos!}
	\note{
		Agradecer especialmente a las familias, al tutor Dr. Sergio Moya, a la
		Facultad de Ingeniería, docentes y compañeros. Este proyecto no hubiera
		sido posible sin su apoyo.
	}
\end{frame}

\begin{frame}[standout]
	¿Preguntas?
\end{frame}

\begin{frame}[allowframebreaks]{Referencias}
	\printbibliography[heading=none]
\end{frame}

% ==============================================================================
% Slides de Backup
% ==============================================================================
\appendix

\begin{frame}{Esquemático del Hardware}
	% TODO: Esquematico KiCad
	\centering
	\fbox{\parbox{0.8\textwidth}{\centering\vspace{3cm}
			[Esquematico KiCad del nodo]
			\vspace{3cm}}}
	\note{
		Backup: Esquemático completo del nodo cámara diseñado en KiCad. Muestra
		conexiones entre ESP32-CAM, sensor PIR, buck converter y baterías.
	}
\end{frame}

\begin{frame}{Planos de la Carcasa}
	% TODO: Planos 2D
	\centering
	\fbox{\parbox{0.8\textwidth}{\centering\vspace{3cm}
			[Planos 2D de la carcasa]
			\vspace{3cm}}}
	\note{
		Backup: Planos 2D de la carcasa con dimensiones exactas. Permite
		fabricación con otros métodos además de impresión 3D.
	}
\end{frame}

\begin{frame}{Análisis Económico}
	\begin{columns}
		\begin{column}{0.5\textwidth}
			\textbf{Inversión inicial:} \$20,530 USD

			\vspace{0.5em}
			\textbf{Modelo de negocio:}
			\begin{itemize}
				\item Cobro por instalación
				\item Suscripción mensual (\$50 USD)
			\end{itemize}

			\vspace{0.5em}
			\textbf{Mercado objetivo:}
			\begin{itemize}
				\item 116 reservas naturales
				\item 10,800+ EAPs con bosques
			\end{itemize}
		\end{column}
		\begin{column}{0.5\textwidth}
			\textbf{Indicadores:}
			\begin{itemize}
				\item VAN: \qty{7517.55}{\$} USD
				\item TIR: \qty{25}{\percent}
				\item TREMA: \qty{15}{\percent}
				\item Recupero: Ano \num{4}
			\end{itemize}

			\vspace{0.5em}
			\textbf{Conclusion:}\\
			Proyecto \textbf{economicamente viable}
		\end{column}
	\end{columns}
	\note{
		Backup: Análisis económico. Inversión inicial de \qty{20530}{\$} USD. Modelo de
		suscripción mensual. VAN positivo, TIR del \qty{25}{\percent} supera TREMA del \qty{15}{\percent}.
		Recupero en año \num{4}. Proyecto económicamente viable.
	}
\end{frame}

\begin{frame}{Sensibilidad Económica}
	% TODO: Graficos de sensibilidad
	\centering
	\fbox{\parbox{0.8\textwidth}{\centering\vspace{2.5cm}
			[Graficos de análisis de sensibilidad]
			\vspace{2.5cm}}}

	\begin{itemize}
		\item Punto de equilibrio: \num{24} suscripciones/ano
		\item Precio minimo viable: \qty{40}{\$} USD/mes
	\end{itemize}
	\note{
		Backup: Análisis de sensibilidad. Punto de equilibrio en \num{24} suscripciones
		anuales. Precio mínimo viable de \qty{40}{\$} USD/mes antes de que el proyecto deje
		de ser rentable.
	}
\end{frame}

\end{document}
